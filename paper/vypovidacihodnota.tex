\section{Vypovídací hodnota sebraných dat}

Než přikročíme k interpretaci sebraných dat, je třeba věnovat
pozornost skutečnostem, které mají vliv na jejich vypovídací
hodnotu.

\subsection{Pravdivost údajů}

Jako u většiny anonymních internetových dotazníků,
nemáme žádnou jistotu, že respondenti skutečně patří do cílové
skupiny. Dotazník mohl vyplnit každý, kdo se o něm dozvěděl.

Nebylo vyvinuto žádné úsilí ve snaze zajistit, aby každý
respondent mohl dotazník vyplnit právě jednou.
Pokud by to tedy někoho bavilo, mohl dotazníků vyplnit
neomezené množství.

Z povahy dotazníku plyne nemožnost ověřit, zda respondenti
uváděli pravdivé informace.

\subsection{Reprezentativnost vzorku}

I pokud by všichni respondenti byli skutečně uživateli Signálů,
vyplnili dotazník každý právě jednou a pravdivě,
je nezbytné ptát se, nakolik je sebraný vzorek reprezentativní.
Jak velkou část aktivních uživatelů se podařilo podchytit?
Jsou ve vzorku zastoupeny pokud možno všechny specifické
podskupiny, nebo došlo k nějaké selekci?

Dotazník cílil především na aktivní uživatele, tedy na ty,
kdo Signály navštěvují denně nebo častěji. Tomu byla uzpůsobená
doba sběru dat, omezená na jeden týden,
a způsob oslovování respondentů, omezující se na
decentní propagaci na titulní stránce.
Pokud je nějaká významnější skupina uživatelů, která na Signály
chodí pravidelně, ale jen týdně nebo řídčeji, je pravděpodobné,
že ve vzorku není příliš zastoupena.

Šetření probíhalo na přelomu července a srpna, tedy v době
dovolených, kdy lze na sociálních sítích předpokládat spíše
nižší aktivitu uživatelů.
Lze předpokládat, že to mělo vliv na \emph{velikost} vzorku,
který by v jiném ročním období snad byl o něco větší.
Výrazný vliv na jeho \emph{složení} nepředpokládám.

K určité selekci respondentů mohlo dojít vzhledem k formě propagace.
Dotazník byl totiž propagován hlavně článkem zařazeným v bloku
redakčního obsahu na titulní stránce. Redakční obsah je sice
na prominentním místě, všem na očích, ale považuji za pravděpodobné,
že mu i značná část uživatelů věnuje pozornost malou až žádnou.

Dalším možným faktorem selekce je osoba průzkum provádějící:
od začátku totiž šlo o soukromý osobní počin. Autor se pak stěží
může považovat za osobu známou a všeobecně oblíbenou.
Je tudíž možné, že jsou ve vzorku nedostatečně zastoupeni
\uv{ti, kdo nemají rádi \suser{dromedar}a} (a tudíž nemají chuť
zapojovat se do jeho podniků).
Není ale důvod myslet si, že by tato skupina, pokud vůbec existuje,
byla nějak velká nebo charakteristicky profilovaná a její případné
nezapojení se mělo zásadnější vliv na celkový obraz.

\subsection{Předpoklad spolehlivosti}

Viděli jsme, že spolehlivost sebraných dat ohrožuje celá řada
faktorů. Na druhou stranu se ale můžeme ptát:
je nějaký důvod domnívat se, že data v některém ohledu nespolehlivá
skutečně jsou?

Nevšiml jsem si ničeho, co by naznačovalo, že někdo dotazník
záměrně vyplnil opakovaně. Nevidím žádný důvod předpokládat
principielní nepravdivost sebraných dat.

Pokud jde o reprezentativnost vzorku, je namístě větší opatrnost.
Pro zpřesnění pohledu by bylo užitečné mít informace
o návštěvnosti webu ve sledovaném období a o skladbě (např. věkové)
aktivních uživatelů, jak je mají k dispozici
správci webu.\footnote{
  Nepamatuji si ovšem, že by data tohoto druhu za celou historii
  Signálů kdy byla zveřejněna, a předpokládám, že se z toho
  spíš nebude dělat výjimka.
}
I zde však snad můžeme předpokládat, že vzorek víceméně
reprezentativní je.

\textbf{Dále tedy budeme předpokládat, že sebraná data jsou v zásadě
pravdivá a představují reprezentativní vzorek uživatelské základny
Signálů.}
