\subsubsection{Náměty na rozvoj signaly.cz}

\setlength{\parskip}{0.3cm}

Uvítala bych vyhledávání dle určitého slova ve společenství na nástěnce (vrácení se k některým příspěvkům)

Vrátil bych signaly.cz do podoby z roku 2008

stránky by se mohly sami aktualizovat

myslím, že by bylo super udělat tu například nějakou miniseznamku nebo nějaké společenství, kde by se lidé mohli seznamovat :)

Mohla by být užitečná otázka, ohledně pomoci, rady od vrstevníka, laika, kněze

"Bylo by sympatické, kdyby signály byly ""trychtýřem"" pro další informační zdroje - veřejné blogy zajímavých osobností i zahraničních, lidí jako Orko Vácha i jiných (ne nutně katolíků), připadají mi v tomto směru prostě spící. Člověk se díky facebooku dozví spoustu informací a bylo by fajn překlopit tuto situaci i do křesťanského prostředí. Možná už se o to v současnosti snažíte, ale buď je těch informačních/ zajímavostních impulzů málo nebo jsou interpretovány nějak špatně, protože mám dojem, že se ke mě nedostávají. Jako je ""výběr z blogů"" líbil by se mi i nějaký ""výběr z venku"".
 A vůbec bych se nezlobila, kdyby se signály trochu převlékly. Za ty roky se ta zelená docela okouká... :(
Držím palce +"

Asi bych netlačila tolik na to, že signály jsou pro mladé. Nemůže to být prostě křesťanský komunitní web? A pak mi přijde líto, že na signálech je spousta hodnotného materiálu, ale schované někde v bažinách společenství. Např. čteme z youcat. Nešlo by tu trochu uklidit a věci dohledatelně porovnat? Díky

Podle mě by nemělo být cílem maximálně rozšiřovat komunitu. Raději umožnit nějak systematicky (podle štítků) roztřídit články; je škoda, když starší články zapadnou a propagují se jen nové.

"Dotazník je úžasný. Jednoduchý, stručnýa  přehledný.
Nebylo by špatné popřemýšlet o tom, aby dnešní signály navázaly na to, kvůli čemu vznikly a na dobu, kdy signály byli skutečně živé.
Některé prvky, které dnes zmizeli by nebylo špatné obnovit (diskuzní společenství; více motivovat mladé k psaní článků na blogy; propojit virtuální signály s realitou - být více vidět na různých akcích, nejen na těch velkých, ale i menších; nabídnout nové možnosti pro starší, kteří odrostli a mají rodiny; zkusit z části navrátit grafiku, aby signály nebyli tolik podobné FB)."

toužím po informovanosti a nebát se jiných názorů, kéž mě signály podpoří v komunikaci či sledování komunikace o závažných tématech co se křesťanů týkají. netrkat hlavu do písku a neplytvat energií

Myslím si, aby měly signály možnost oslovit stávající mladé křesťany, musí se to dít ve spolupráci s faráři, nebo alespoň kaplany pro mládež, které je nutno namotivovat tak, aby si akci vzali za své - taktéž biskupové, od nichž by měl popud vyjít směrem k farářům - a také s řády pracujícími s mládeží (Salesiáni, apod.), investovat do propagace ve farnostech a na školách - např. skrz vytvoření výukového programu o náboženství, filozofii, humanitních vědách, apod., který bude přijatelný pro školy - edukativní složka, která přiměje mladé (i ty nevěřící hledající) na signály přijít a případně se zapojit i jinak. Což lze zase ve spolupráci s katechety, či studenty vysokých škol těchto oborů. Jsem tedy velký podporovatel e-learningu (humanitní vědy, náboženství, osvěta, apod.), zajištěné výlepy plakátů do měst (nejen farností, ale samozřejmě ve spolupráci s farností a dohodnutých fin. podmínek), nebo se zamyslet i nad reklamou v novinách - je to vyšší částka, ale okolo církevních svátků, apod. si lidé kladou různé otázky - udělat si marketingovou strategii a říct si, jestli chceme evangelizovat, nebo jen chytat v našich vodách. Obávám se, že ti akční jedinci, kterých je nedostatek na signálech, jsou právě tak akční, že už nic jiného nestíhají a nevidí ve webu nic hmatatelného. Vidím to sama na sobě, nemám u práce a jiných bohulibých aktivit na nic dalšího čas, tak i můj blog skomírá prázdnotou, ale i na druhou stranu se potýkám s tím, že kdybych chtěla napsat i něco jiného, co má sice říz, ale není to vyloženě křesťanského tématu, tak by to stejně nikoho nezajímalo, protože se na signálech nepsaným pravidlem očekává, že téma bude zbožné, tudíž na "nezbožné" nebo ne primárně zbožné úvahy bych si stejně musela hledat jinou komunitu, nebo mi to za tu tvorbu ani v únavě všedního dne nakonec nestojí - nejsem motivována. Atd. Sur sum corda!

\subsubsection{Výrazy náklonnosti k signaly.cz}

Signály potřebuji a děkuji všem za jejich práci pro nás.

Signály jsou super.

Signály jsou skvělá věc.

Signály jsou nejlepší sociální síť!

Signály jsou bezpečný prostor pro lidi věřící. Věřím, že tady ještě dlouho budou. Díky za ně.

Signálům fandím. Myslím, že i přes všechny problémy má tenhle projekt v nějaké podobě smysl. (pro starší i mladší)

jsem rada,ze jsou...

Jsem ráda, že Signály existují. Snad si na CSM najdu přátelé, se kterými budu přes Signály zažívat jen hezké situace a plánování akcí. ☺

Děkuji, že signály jsou a fungují :)

Děkuji za signály

Doufám, že Signály přetrvají, je tu třeba mít křesťanskou sociální sit...

díky za signály ;-)

Díky Bohu za Signály, jsem velmi potěšená, že mohu chodit na křesťanské stránky, velmi si jich vážím a fandím jim! Díky všem, kteří se na nich podílejí, Pán Bůh zaplať všem!

Signaly jsou skvely, doufam ze se jim bude darit i nadale..:-) Ze Realizacni tym udela signaly.cz jeste atraktivnejsi, aby pomohli treba nezadanym krestanum z ruznych denominaci..

Děkuji za dotazník. Ráda bych, kdyby signály zůstaly víceméně takové jaké jsou. V případě, že se zavede něco jako všeobecná cenzura "nepříjmných komentářů a názorů" asi bych ztratila zájem je více využívat, protože se domnívám že by tímto "sítem" propadla většina mých oblíbených diskutujících, což by byla podle mě škoda.

\subsubsection{K otázce poslání signaly.cz}

Velmi se mi libi pestrost lidi, starsi intelektualove a konzervativci, mlade poeticke zeny, jeste "svezi" mladeznici, je to nejen krestansky unikatni, ale celkove unikatni a diky tomu hodnotna socialni sit, ktera by nemela chybet!

Signály jsou pro mě ze všeho nejvíc blogová platforma. Popravdě, místní společenství mi vůbec nic neříkají a ta, který by snad mohla, jsou mrtvá.

signály jsou podle mě ideální prostor pro svědectví o osobním vztahu s Bohem, o Boží lásce, to je podle mě hlavní přidaná hodnota oproti jiným sociálním sítím

Mám dojem, že na rozdíl od např. FB signály lépe umožňují diskuse lidí mimo vlastní sociální bublinu (kromě křesťanství, ale to je takový vstupní předpoklad...)

ŽMM má obrovský potenciál, matky na mateřské jsou nejaktivnější na internetu. Matka a manželka sice ještě nejsem, ale i tak je to pro mě velmi obohacující a zajímavé....

\subsubsection{Proč už pro mě osobně signaly.cz nejsou tak hodnotné jako dřív}

Změna Signálů podle vzoru Facebooku byla chyba. Mnohem větší chybou však byla perzekuce a vyštvání tradicionalistů a jiných "potížistů".
z mladších generací je tu málo uživatelů

signály.cz pro mě měly velkou důležitost před 5-10 lety, kdy jsem jako mladá pořádala akce a signály nabízely dobrou možnost propagace a také poskytovaly velkou nabídku dalších akcí, kterých jsme se se spolčem účastnili. Také signály.cz sloužily jako dobrý komunikační prostor pro spolčo (vlastní skupina) a prostor pro komunikaci s přáteli. Dnes se mí přátelé přesunuli ve velké míře na facebook a signály v tomto směru ztratily svou atraktivitu. A mladí z mého okolí nemají o signálech ani tušení. Já vám chci vyjádřit osobní díky za vaši práci při zdokonalování webu :)

Na dřívější podobě signálu měl člověk trochu "domovský pocit" v této podobě se to nějak ztrácí. Ale je tam i dost dobrých věcí, takže celkově - je dobré, že jsou.

Jedním z mála důvodů, proč jsem občas na Signály zavítala, byl blog Toba (Karla Skočovského). Kam zmizel? Věčná škoda všech jeho článků, které si už nikdo nepřečte!

"Chtěl bych dodat, že podle mého názoru se signály postupem času až příliš připodobnily facebooku, se kterým však nemohou konkurovat. Proto se samy postavily do role ""slabšího konkurenta"". Drtivá většina lidí na signálech facebook užívá, takže je si toho vědoma rovněž. Uvítal bych, kdyby signály udělaly trochu ""odklon"" od mainstreamu sociálních sítí a šly více vlastní cestou.
I já osobně jsem mnohem častěji a radši chodil na signály, když vypadaly svébytně a nebyly křesťanským ""klonem"" facebooku. Od doby, kdy proběhly aktualizace a změna vzhledu jsem u sebe vypozoroval, že na signály chodím čím dál méně často a čím dál méně je používám. Rovněž méně často píši na blog. Úplně mi vystačí facebook kde stejně všichni známí jsou."

\subsubsection{K "nepříjemným zkušenostem" na signaly.cz}

"Na signálech (ale i obecně v životě) mi chybí větší důraz na pravidla komunikace. Ta virtuální má svá specifika (tvrdí se, že cca 80% komunikace probíhá na neverbální bázi, která je přes síť nezachytitelná) a mým názorem je, že 99% ""nepříjemných pocitů"" vzniká právě z neznalosti těchto specifik.
Druhou otázkou je ""církevní nauka"". Myslím, že zde často dochází ke zmatení. Ale naprosto netuším, jak by se to dalo ""korigovat"". Snad asi vedením uživatelů k tomu, aby si věci, které jsou pro ně důležité, ověřovali v jiných pramenech (jakých, kde apod.)"

"Chci poděkovat za mnohahodinovou práci věnovanou tomuto dotazníku. Vážím si toho. Signály jsou totiž mojí jedinou sociální sítí, kde komunikuju s lidma.
K situaci kolem komentářů: ano, zažila jsem nepříjemné situace, chtěla jsem ze Signálů odejít. Dala jsem si malou pauzu a pak využila možností, které jako uživatel mám - zavřít komentáře pod mými články (a tím se zbavit i pozitivní zpětné vazby), nečíst košaté diskuze, které mě přestaly obohacovat... A ejhle - pozitivní zpětné vazby začaly přicházet spontánně formou soukromých zpráv.
Všimla jsem si, že se občas na Signálech vyskytují anonymní profily, jejichž cílem je vyloženě provokovat, urážet. Dokonce rýpavým způsobem shazovat autoritu papeže. Myslím si, že by s tím realizační tým měl něco dělat. Pokud budu mít čas, zkusím dohledat konkrétní příklad. "

\subsubsection{Reakce neregistrovaných uživatelů (!)}

Nejsem na Signálech registrovaná a ani to neplánuju, ale "chodím" na ně celkem často a záleží mi na nich.

signaly.cz sleduji minimálně už 6 let, poslední 3 roky intenzivněji (denně), dosud jsem si ale nezaložila vlastní profil

\subsubsection{Věcná zpětná vazba k dotazníku}

tenhle dotazník je dost dlouhej

Dobry napad, libi se mi i provedeni, nezaznamenala jsem sugestivni otazky :)

Děkuji všem, kteří práci na Signálech věnují svůj čas!
část dotazníku, kde se vyplňuje, co vše jsem na signály.cz zažil (jel jsem na akci atd.) není moc relevantní, neboť není časově vymezená např. v posledním roce -protože mnohé zažité věci jsou (jistě nejen u mě) z dřívější aktivnější éry signály.cz.

\subsubsection{Upřesnění k odpovědím na strukturované otázky}

Ohledne těch nepříjemných vzkazu/komentaru berte to s rezervou je to tak šest let zpátky.

Kdyz mam po maturite, tak nevim, co mam dat, ze studuji. Ale to je ted u vsech dotazniku :D.

\subsubsection{Pozdravy, díky a jiné méně věcné odpovědi}

velmi pěkný dotazník těším se na vyhodnocení

těším se na výsledky. díky

Tento dotazník byl výborný nápad! Díky.

S Boží pomocí, je to na Něm, jak to bude!

přeji pěkný den :-)

Jen pokračujte v Božím díle +

Good job!

Díky =)

Děkuju za dotazník :)

Díky za práci se sestavením dotazníku.

Díky za dotazník, těším se na vyhodnocení.

Díky za vaše snahy o růst a zvyšování kvality Signálů...

Díky za tento dotazník, snad přinese své plody!

Dotazník je super, dík že sis s ním dal tu práci. Doufám, že z něho vzejde něco zajímavého. Anonym

Hotovo, snad Vám to nějak pomůže. Děkuji za příjemný dotazník. Vyprošuji Vám Boží požehnání +

Díky borci, který má zájem o signály a jejich budoucnost za vytvoření dotazníku ;) jsem zvědav, co se bude dít. Veni Sancte Spiritus :)

\subsubsection{Jiné}

Signály.cz by měly rozjet marketing, který nakopne uživatele k jejich většímu používání.

signály jako takové mi přijdou super. úplně jde cítit, že nejsou facebookem apod, je tam úplně jiné prostředí, jiný duch, lidé tam nereagují tak jako reagují lidé ve skupinách na fb, kterých jsem kdysi byla součástí. nemám nic proti signálům, jako takové je velmi podporuji a říkám o nich ostatním a zvu je atd, zvláště ty hledající křestanské prostředí, kterého se jim nedostává ve farnosti, mezi ostatními maminkami na mateřské z okolí atp.... ale není to místo pro mě. mám pocit, jako kdyby mě Pán chtěl jinde. nechce, abych se upínala k nějakému společenství, lidem, názorem... chce abych se upnula jen k Němu. někdy mi je tak strašně těžko na duši, když je používám... tedy když to shrnu, signály jsou dle mě super, nápad, vedení i uživatelé, ale nejsou na mé životní či duchovní cestě či jak to nazvat.

signály byly dříve fajn, teď mi to přijde jako mrtvej projekt... ale díky moc lidem, kteří jej zachraňují, myslím však, že obsah by měly především tvořit uživatelé

Bohužel jsem zatím nepřišel na nějaké větší výhody signálů. S kamarády máme názor, že signály nemají větší potenciál.

\setlength{\parskip}{0cm}
