\documentclass[12pt, a4paper, twoside]{article}

\usepackage{fontspec}
\setmainfont[Ligatures={TeX}]{TeXGyrePagella}

\usepackage[czech]{babel}
\usepackage{csquotes}
\usepackage[left=2.5cm, right=2.5cm, top=2.5cm, bottom=2.5cm, marginparsep=3mm]{geometry}

\usepackage[
  backend=biber,
  style=iso-authoryear,
  sortlocale=cs_CZ,
  maxnames=3,
  firstinits=true,
]{biblatex}

\usepackage[show]{ed} % editorial annotations
\usepackage{color}
\usepackage{xcolor}
\usepackage[hidelinks]{hyperref}
\usepackage{changepage}
\usepackage{nameref}
\usepackage{graphicx}
\usepackage{float} % necessary for figures with placement [H]

\newcommand{\suser}[1]{\href{https://www.signaly.cz/#1}{\texttt{@#1}}}

\newcommand{\answercount}[1]{Odpovědělo \input{graphs/counts/#1.txt} respondentů}

\newcommand{\includegraph}[2]{
  \begin{figure}[H]
    \centering
    \textbf{#2}
    \includegraphics{graphs/img/#1.pdf}
    \answercount{#1}
  \end{figure}
}


\setcounter{secnumdepth}{3}
\setcounter{tocdepth}{3}

\extrafloats{200} % greater limit of floats awaiting placement

\bibliography{biblio}

\author{Jakub Pavlík}
\title{Vyhodnocení dotazníku\\ \uv{signaly.cz z pohledu uživatelů}}

\begin{document}

\setlength{\parindent}{0.5cm}

\maketitle

\section*{Úvod}

Signály letos slaví deset let provozu jako sociální síť.
Kromě toho, že se slaví a vzpomíná, však naléhavě vyvstávají otázky
ohledně dalšího směřování celého projektu:
není třeba provoz na Signálech pozorovat zvlášť dlouho ani pozorně,
aby pozorovatel pochopil, že heslo z titulní stránky
\emph{33 637 mladých křesťanů skrze signály.cz tvoří společenství}
se s realitou hrubě rozchází:
web sice snad má přes 33000 registrovaných
uživatelů, převážná většina uživatelských účtů však je buďto
úplně opuštěná, nebo vykazuje jen minimální aktivitu.
Navíc se ukazuje -- jak bude vidět i dále ve výstupech z dotazníku --
že uživatelskou základnu spíše než mladí křesťané
tvoří \emph{křesťané, kteří v době spuštění Signálů jako komunitního
  webu byli mladí:}
uživatelská základna stárne, Signály nejsou příliš úspěšné
v oslovování nových uživatelů z řad mládeže.
Mezi deklarovaným posláním a tím, kým a k čemu jsou Signály skutečně
využívány, zeje propast, která se nadále rozevírá.

Tyto skutečnosti si žádají nové definování vize a poslání Signálů.
Přispět k tomu jsem se pokusil s pomocí dotazníku, jehož vyhodnocení
překládám. V základech tohoto podniku stojí předpoklad,
který jsem formuloval již v úvodu k dotazníku:
\emph{hlavní hodnotou sociální sítě jsou její uživatelé --
  a cesta k uživatelům novým vede přes pochopení potřeb těch
  stávajících}.

Šetření probíhalo v týdnu 26.~7.--2.~8. 2017
s pomocí dotazníku postaveného v Google Forms.
K oslovení respondentů byl využit článek na osobním blogu
\url{http://dromedar.signaly.cz}, redakcí Signálů pro ten účel
na celý týden laskavě vystavený v bloku redakčního obsahu
na titulní stránce, takže se dostal i k těm, kdo daný blog normálně
nesledují.
Za týden, kdy byl dotazník propagován a probíhal sběr odpovědí,
se jich podařilo nasbírat 215. Je potřeba zohlednit to, že
šetření probíhalo v době dovolených, po poměrně krátkou dobu
a bez masivnější \uv{marketingové podpory}.
I přesto počet sebraných odpovědí jistě má určitou vypovídací
hodnotu ohledně velikosti aktivní části uživatelské základny.

\section{Otázky}

Otázky byly rozděleny do šesti bloků.
První blok (\nameref{sec:mojeaktivita})
je nepříliš sourodá skupina otázek bez opravdového společného tématu.

Další tři
(\nameref{sec:kcemu}, \nameref{sec:funkcionality}, \nameref{sec:jinesite})
se z různých stran snaží dobýt odpověď na otázku, čím Signály jsou
pro své uživatele, jako co jsou používány.
Opovídají priority uživatelů prioritám redakce a vývoje?
A pokud ne, je to mylnými představami o prioritách uživatelů,
nebo třeba snahou oslovit potenciální uživatele s jinými prioritami?

Předposlední blok otázek (\nameref{sec:ostatniuzivatele})
reaguje na hlasy, že na Signálech je nevlídné prostředí
a je třeba ho \uv{zútulnit} propracovanějším systémem nastavení
soukromí, nebo postihy uživatelů, kteří jsou vnímáni jako problémoví.
Odpovědi na otázky tohoto bloku by měly ukázat, jak velká část
uživatelské základny tyto problémy vnímá jako palčivé.

Otázky závěrečného bloku (\nameref{sec:osobni})
slouží ke strukturování odpovědí
podle vybraných antropologických proměnných.

\vfill

\subsection{Moje aktivita na Signálech}\label{sec:mojeaktivita}

\subsubsection{Na Signálech mám profil}

\includegraph{na_signalech_mam_profil}{}

\subsubsection{O Signálech jsem se poprvé dozvěděl/a}

\includegraph{o_signalech_jsem_se_poprve_dozvedel}{}

\subsubsection{Na Signály chodím}

\includegraph{na_signaly_chodim}{}

\subsection{K čemu Signály používám}\label{sec:kcemu}

\subsubsection{Na Signály chodím hlavně}

\includegraph{na_signaly_chodim_hlavne}{}

\subsubsection{Signály jsem využil/a k}

\includegraph{signaly_jsem_vyuzil_k}{}

\subsubsection{Na Signálech \uv{se mi povedlo}}

\includegraph{na_signalech_se_mi_povedlo}{}

\subsection{Funkcionality, které využívám}\label{sec:funkcionality}

\subsubsection{Jsem autorizovaný/á}

\includegraph{jsem_autorizovany}{}

\subsubsection{Jak často využívám jednotlivé funkcionality}

\includegraph{funkcionality_status_pisu}{Na svou zeď píšu}

\includegraph{funkcionality_status_ctu}{Statusy přátel čtu}

\includegraph{funkcionality_status_komentuji}{Statusy přátel komentuji}

\includegraph{funkcionality_sledovane_blogy_ctu}{Svoje sledované blogy čtu}

\includegraph{funkcionality_blogy_vyber_ctu}{Výběr z blogů čtu}

\includegraph{funkcionality_blogy_nesledovane_ctu}{Blogy, které nesleduji, ani nejsou ve výběru, čtu}

\includegraph{funkcionality_blog_komentuji}{Komentář pod nějaký blog píšu}

\includegraph{funkcionality_blog_pisu}{Příspěvek na blog píšu}

\includegraph{funkcionality_fotky_prohlizim}{Fotky si prohlížím}

\includegraph{funkcionality_fotky_vkladam}{Fotky vkládám}

\includegraph{funkcionality_akce_prohlizim}{Nabídku akcí si prohlížím nebo hledám akci, kam pojedu}

\includegraph{funkcionality_akce_vkladam}{Vkládám akci}

\includegraph{funkcionality_zed_spolecenstvi_ctu}{Zeď společenství čtu}

\includegraph{funkcionality_zed_spolecenstvi_pisu}{Na zeď společenství píšu}

\includegraph{funkcionality_videa_prohlizim}{Na videa se dívám}

\includegraph{funkcionality_videa_vkladam}{Videa vkládám}

\includegraph{funkcionality_autorizovani_chat}{(Vyplňují jen autorizovaní uživatelé) Chat používám}

\subsection{Signály a jiné sociální sítě}\label{sec:jinesite}

\subsubsection{Používáš kromě Signálů i jiné sociální sítě? Jak často navštěvuješ}

\includegraph{jine_site_Facebook}{Facebook}

\includegraph{jine_site_Twitter}{Twitter}

\includegraph{jine_site_Youtube}{Youtube}

\includegraph{jine_site_Instagram}{Instagram}

\includegraph{jine_site_Googleplus}{Google+}

\includegraph{jine_site_Lidecz}{Lidé.cz}

\subsubsection{Používáš další sociální sítě?}

\includegraph{pouzivas_dalsi_socialni_site}{}

\subsubsection{Proč používáš Signály, i když je dnes bohatá nabídka jiných sociálních sítí?}

\includegraph{proc_signaly_jenom_signaly}{Používám jenom Signály, jiné sociální sítě nepotřebuji}

\includegraph{proc_signaly_signaly_nejdulezitejsi}{Signály jsou pro mě nejdůležitější sociální síť}

\includegraph{proc_signaly_kamaradi_kteri_jinde_nejsou}{Na Signálech jsem kvůli kamarádům, kteří na jiných sociálních sítích nejsou}

\includegraph{proc_signaly_zajimavi_lide_kteri_jinde_nejsou}{Na Signálech jsem kvůli zajímavým lidem, kteří nejsou na jiných sociálních sítích}

\includegraph{proc_signaly_jedinecna_aktivita}{Na Signálech jsem kvůli něčemu, co se děje jenom tam (např. aktivita nějakého společenství)}

\includegraph{proc_signaly_jedinecne_informace}{Na Signálech jsem kvůli informacím, které jinde nejsou}

\includegraph{proc_signaly_oblibene_blogy}{Na Signálech jsem kvůli blogům, které rád čtu}

\includegraph{proc_signaly_krestanske_prostredi}{Na Signálech jsem proto, že mám rád křesťanské prostředí}

\subsubsection{Pokud jseš na Signálech (ještě) kvůli něčemu jinému, můžeš nám to prozradit tady: (volná otázka)}

\answercount{proc_signaly_jine}.
Odpovědi byly částečně uspořádány podle témat.

\begin{itemize}
\item Hlavně jsem chtěl někde založit blog. Křesťanská sociální síť se zdála být vhodným prostředím :)

\item mam tu blog

\item Kvůli psaní blogu.

\item zvláštní směsice zábavy  a poučení v "jazyce", kterému rozumím

\item Určitou roli v tom hraje setrvacnost na signály jsem chodil ještě v době kdy nebyly sociální síť... signály znám cca. 14 let byl jsem i zapojen do redakce starých signalu

\item Setrvačnost, zvyk

\item ze setrvačnosti

\item ze setrvačnosti

\item částečně ze stereotypu - když už jsem se kdysi registrovala a začala psát blog, občas se tam prostě kouknu.

\item z nostalgie:)

\item Z nostalgie.

\item z nostalgie

\item z nostalgie, z věrnosti

\item Připadají mi mnohem smysluplnější než fb a jiné sociální sítě...

\item žmm

\item kvůli společenství ŽMM

\item Kvůli společenství ŽMM, jinak je tu mrtvo a nechodila bych sem

\item dialog (nejen) s mladou generací

\item Povzbuzují mě vybraná videa

\item Je to prostor pro diskusi mezi křesťany, jaký na českém internetu nemá konkurenci.

\item Na signálech jsem proto, že mě přibližují Bohu a pomáhají mi lépe žít křesťanským životem.

\item nacházím zde inspiraci, jak obohatit farnost, ve které žiji

\item Se zájmem sleduji diskuse různých uživatelů a z nějakého důvodu nejraději právě těch, kteří jsou některými vnímáni jako "šťouralové", přestože ne vždy s nimi souhlasím. Líbí se mi, že dokážou svými příspěvky aspoň některé (například mě) přivést k zamyšlení, někdy k rozšíření obzorů, někdy k přehodnocení stanovisek a někdy třeba jen k utvrzení toho co si už myslím. Sama se diskusí aktivně zúčastňuji jen zřídka, poslední dobou spíše jen lajknutím komentáře, který se mi zdá trefný.

\item Obsah, který zveřejním, vidí pouze věřící známí. Na fb ho vidí i lidé z nevěřícího prostředí, kterým by mohlo sem tam něco připadat dost divné. A ani mě by nebylo příjemné, aby si moje "nábožné" statusy četli.

\item Signály jsou více komornějšía  rodinější, než jiné sociální sítě. Používám je i třeba ke zpětné reakci či vzpomínání na proběhlé akce. Ke komunikaci s uživateli, které mohu potkat osobně.

\item Kvůli maminkovské starosti i radosti

\item Na signálech jsem kvůli křesťanskému prostředí které na jiných soc. síť. není

\item Pro mě jsou signály.cz bezpečnou soc. sítí: bez nepříjemných až sexuálně motivovaných reklam (sponzorované příspěvky na FB), autorizovaní uživatelé (skvělá prevence anonymity).

\item pro povzbuzení

\item práce

\item inspirace k duchovnímu růstu. taky tady hledám vtipy vhodné do farního časopisu, recenze knížek.

\item Vkládám jsem fotky

\item dřív to bylo kvůli společenství křesťanských uživatelů, četla jsem povzbuzující a zajímavé články od přátel, jezdila jsem na akce, kde signály byly a potkávala lidi z týmu... dnes veškerá aktivita klesla, možná je to tím, že lidi, kteří byli aktivní v té době už teď mají své rodiny a jsou aktivnější v reálu a mladší kamarádi jsou na facebooku

\item Signály jsou pro mě určitá náhrada facebooku. Znám se a vím, že kdybych měla facebook, tak jsem na něm hodně často a šťourala bych se ve věcech, který by mi nebyly k prospěchu. Signály jsou v tomto bezpečnější, protože tady není tak velká aktivita, takže nezabiju tolik času - za 15 minut mám po prohlížení :D A navíc sem lidi nedávají věci, který bych o nich radši nevěděla, jak je tomu jinde.

\item jen k tomu posledním - křesťanské prostředí..  trochu:-). Také ta starší podoba signálu byla myslím lepší.("Modernější" není v každém případě lepší)

\item je to křesťanstská sociální síť
\end{itemize}

\subsection{Já a ostatní uživatelé}\label{sec:ostatniuzivatele}

\includegraph{ostatni_tesi_me_libi_se_kamaradi}{Těší mě, když se mým kamarádům líbí můj obsah (status, fotka, blog, video)}

\includegraph{ostatni_tesi_me_libi_se_cizi}{Těší mě, když se cizím lidem líbí můj obsah}

\includegraph{ostatni_tesi_me_komentar_kamaradi}{Těší mě, když moji kamarádi komentují můj obsah}

\includegraph{ostatni_tesi_me_komentar_cizi}{Těší mě, když cizí lidé komentují můj obsah}

\includegraph{ostatni_tesi_me_zprava_kamaradi}{Těší mě, když mi kamarádi píšou soukromé zprávy}

\includegraph{ostatni_tesi_me_zprava_cizi}{Těší mě, když mi cizí lidé píšou soukromé zprávy}

\includegraph{ostatni_autorizovani_duveryhodnejsi}{Autorizovaní uživatelé jsou pro mě důvěryhodnější než neautorizovaní}

\includegraph{ostatni_neprijemne_komunikace_obtezuje}{Komunikace na Signálech mě někdy obtěžuje, nebo je mi z ní těžko}

\includegraph{ostatni_neprijemne_neprijemne_situace}{Na Signálech jsem zažil/a situace, které jsou mi nepříjemné}

\includegraph{ostatni_neprijemne_zazivat_nechci}{Na Signálech jsem zažil/a situace, které zažívat nechci}

\subsubsection{Na Signálech jsem zažil/a}

\includegraph{neprijemne_co}{}

\subsubsection{(Jen ti, kdo zažili nepříjemné komentáře) Komentáře mi byly nepříjemné, protože}

\includegraph{neprijemne_komentare_protoze}{}

\subsubsection{(Jen ti, kdo zažili nepříjemné komentáře)}

\includegraph{neprijemne_komentare_zvazoval_odchod}{Uvažoval/a jsem, že kvůli nepříjemným komentářům ze Signálů odejdu}

\includegraph{neprijemne_komentare_zakazat}{Na Signálech by mělo být zakázané psát takové komentáře}

\includegraph{neprijemne_komentare_pisatele_pryc}{Ti, kdo píší takové komentáře, by na Signálech vůbec neměli být}

\subsubsection{(Jen ti, kdo zažili nepříjemné vzkazy) Vzkazy mi byly nepříjemné, protože}

\includegraph{neprijemne_vzkazy_protoze}{}

\subsubsection{(Jen ti, kdo zažili nepříjemné vzkazy)}

\includegraph{neprijemne_vzkazy_zvazoval_odchod}{Uvažoval/a jsem, že kvůli nepříjemným vzkazům ze Signálů odejdu}

\includegraph{neprijemne_vzkazy_zakazat}{Na Signálech by mělo být zakázané psát takové vzkazy}

\includegraph{neprijemne_vzkazy_pisatele_pryc}{Ti, kdo píší takové vzkazy, by na Signálech vůbec neměli být}

\subsection{Osobní údaje}\label{sec:osobni}

\subsubsection{Jsem}

\includegraph{jsem}{}

\subsubsection{Věk}

\includegraph{vek}{}

\subsubsection{Studuji}

\includegraph{studuji}{}

\subsubsection{(Jen žáci a studenti) Jsem žák / student}

\includegraph{studuji_co}{}

\subsubsection{Moje nejvyšší dosažené vzdělání}

\includegraph{moje_nejvyssi_dosazene_vzdelani}{}

\section{Volná otázka \uv{Chci dodat}}

\section{Vypovídací hodnota sebraných dat}

(Ne)Ochrana před vícenásobným vyplněním

Nepravdivé informace

Reprezentativnost vzorku vzhledem k počtu aktivních uživatelů

Možná selekce uživatelů (ti, které vůbec nezajímá redakční obsah,
nemají rádi autora dotazníku apod.)

\section*{Poděkování}

Závěrem je třeba poděkovat všem, kdo se na dotazníku podíleli:
\suser{psycho-kat}, jejíž otázka po \uv{tvrdých datech}
vznik dotazníku vyprovokovala;
\suser{JiKu}, \suser{slu-nicko}, \suser{plihalik}, \suser{mia-maru},
\suser{Papo} a \suser{alweryon}, kteří pročetli otázky před spuštěním
dotazníku a poskytli mi k nim zpětnou vazbu;
šéfredaktorce \suser{Kollenka}, která článek s pozváním k vyplnění
dotazníku na celý týden umístila do bloku redakčního obsahu,
takže se dostal
k mnohem více uživatelům, než by běžný příspěvek z osobního blogu
kdy mohl;
zejména ale patří velký dík každému, kdo věnoval kus svého času
a dotazník vyplnil.

\tableofcontents

\printbibliography

\end{document}
