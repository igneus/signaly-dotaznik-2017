\documentclass[12pt, a4paper, twoside]{article}

% To compile:
%
% $ xelatex filename
% $ biber filename
% $ xelatex filename

\usepackage{fontspec}
\setmainfont[Ligatures={TeX}]{TeXGyrePagella}

\usepackage[czech]{babel}
\usepackage{csquotes}
\usepackage[left=2.5cm, right=2.5cm, top=2.5cm, bottom=2.5cm, marginparsep=3mm]{geometry}

\usepackage[
  backend=biber,
  style=iso-authoryear,
  sortlocale=cs_CZ,
  maxnames=3,
  firstinits=true,
]{biblatex}

\usepackage[show]{ed} % editorial annotations
\usepackage{color}
\usepackage{xcolor}
\usepackage[hidelinks]{hyperref}
\usepackage{changepage}
\usepackage{nameref}
\usepackage{graphicx}

\newcommand{\suser}[1]{\href{https://www.signaly.cz/#1}{\texttt{@#1}}}


\setcounter{secnumdepth}{3}
\setcounter{tocdepth}{3}

\bibliography{biblio}

\author{Jakub Pavlík}
\title{Vyhodnocení dotazníku\\ \uv{signaly.cz z pohledu uživatelů}}

\begin{document}

\setlength{\parindent}{0.5cm}

\maketitle

\section*{Úvod}

Signály letos slaví deset let provozu jako sociální síť.
Kromě toho, že se slaví a vzpomíná, však naléhavě vyvstávají otázky
ohledně dalšího směřování celého projektu:
není třeba provoz na Signálech pozorovat zvlášť dlouho ani pozorně,
aby pozorovatel pochopil, že heslo z titulní stránky
\emph{33 637 mladých křesťanů skrze signály.cz tvoří společenství}
se s realitou hrubě rozchází:
web sice snad má přes 33000 registrovaných
uživatelů, převážná většina uživatelských účtů však je buďto
úplně opuštěná, nebo vykazuje jen minimální aktivitu.
Navíc se ukazuje -- jak bude vidět i dále ve výstupech z dotazníku --
že uživatelskou základnu spíše než mladí křesťané
tvoří \emph{křesťané, kteří v době spuštění Signálů jako komunitního
  webu byli mladí:}
uživatelská základna stárne, Signály nejsou příliš úspěšné
v oslovování nových uživatelů z řad mládeže.
Mezi deklarovaným posláním a tím, kým a k čemu jsou Signály skutečně
využívány, zeje propast, která se nadále rozevírá.

Tyto skutečnosti si žádají nové definování vize a poslání Signálů.
Přispět k tomu jsem se pokusil s pomocí dotazníku, jehož vyhodnocení
překládám.

Dotazníkové šetření probíhalo v týdnu 26.~7.--2.~8. 2017
s pomocí dotazníku postaveného v Google Forms.
K oslovení respondentů byl využit článek na osobním blogu
\url{http://dromedar.signaly.cz}, redakcí Signálů pro ten účel
na celý týden laskavě vystavený v bloku redakčního obsahu
na titulní stránce, takže se dostal i k těm, kdo blog normálně
nečtou.
Za týden, kdy byl dotazník propagován a probíhal sběr odpovědí,
se jich podařilo nasbírat 215. Je potřeba zohlednit to, že
šetření probíhalo v době dovolených, po poměrně krátkou dobu
a bez masivní \uv{marketingové podpory},
přesto počet sebraných odpovědí jistě má i určitou vypovídací
hodnotu ohledně velikosti aktivní části uživatelské základny.

\section{Otázky}

Otázky byly rozděleny do šesti bloků.
První blok (\nameref{sec:mojeaktivita})
je nepříliš sourodá skupina otázek bez opravdového společného tématu.

Další tři
(\nameref{sec:kcemu}, \nameref{sec:funkcionality}, \nameref{sec:jinesite})
se z různých stran snaží dobýt odpověď na otázku, čím Signály jsou
pro své uživatele, jako co jsou používány.
Opovídají priority uživatelů prioritám redakce a vývoje?
A pokud ne, je to mylnými představami o prioritách uživatelů,
nebo třeba snahou oslovit potenciální uživatele s jinými prioritami?

Předposlední blok otázek (\nameref{sec:ostatniuzivatele})
reaguje na hlasy, že na Signálech je nevlídné prostředí
a je třeba ho \uv{zútulnit} propracovanějším systémem nastavení
soukromí, nebo postihy uživatelů, kteří jsou vnímáni jako problémoví.
Odpovědi na otázky tohoto bloku by měly ukázat, jak velká část
uživatelské základny tyto problémy vnímá jako palčivé.

Otázky závěrečného bloku (\nameref{sec:osobni})
slouží ke strukturování odpovědí
podle vybraných antropologických proměnných.

\subsection{Moje aktivita na Signálech}\label{sec:mojeaktivita}

\subsubsection{Na Signálech mám profil}

\includegraphics{graphs/img/na_signalech_mam_profil.png}

\subsubsection{O Signálech jsem se poprvé dozvěděl/a}

\includegraphics{graphs/img/o_signalech_jsem_se_poprve_dozvedel.png}

\subsubsection{Na Signály chodím}

\includegraphics{graphs/img/na_signaly_chodim.png}

\subsection{K čemu Signály používám}\label{sec:kcemu}

\subsubsection{Na Signály chodím hlavně}

\includegraphics{graphs/img/na_signaly_chodim_hlavne.png}

\subsubsection{Signály jsem využil/a k}

\includegraphics{graphs/img/signaly_jsem_vyuzil_k.png}

\subsubsection{Na Signálech \uv{se mi povedlo}}

\includegraphics{graphs/img/na_signalech_se_mi_povedlo.png}

\subsection{Funkcionality, které využívám}\label{sec:funkcionality}

\subsubsection{Jsem autorizovaný/á}

\includegraphics{graphs/img/jsem_autorizovany.png}

\subsubsection{Jak často využívám jednotlivé funkcionality}

+ jen autorizovaní: chat

\subsection{Signály a jiné sociální sítě}\label{sec:jinesite}

\subsubsection{Používáš kromě Signálů i jiné sociální sítě? Jak často navštěvuješ}

\subsubsection{Používáš další sociální sítě?}

\includegraphics{graphs/img/pouzivas_dalsi_socialni_site.png}

\subsubsection{Proč používáš Signály, i když je dnes bohatá nabídka jiných sociálních sítí?}

\subsubsection{Pokud jseš na Signálech (ještě) kvůli něčemu jinému, můžeš nám to prozradit tady:}

\subsection{Já a ostatní uživatelé}\label{sec:ostatniuzivatele}

\subsubsection{Těší mě ...}

\subsubsection{Na Signálech jsem zažil/a}

\includegraphics{graphs/img/na_signalech_jsem_zazil.png}

\subsubsection{(Jen ti, kdo zažili nepříjemné komentáře) Komentáře mi byly nepříjemné, protože}

\subsubsection{(Jen ti, kdo zažili nepříjemné komentáře)}

\subsubsection{(Jen ti, kdo zažili nepříjemné vzkazy) Vzkazy mi byly nepříjemné, protože}

\subsubsection{(Jen ti, kdo zažili nepříjemné vzkazy)}

\subsection{Osobní údaje}\label{sec:osobni}

\subsubsection{Jsem}

\includegraphics{graphs/img/jsem.png}

\subsubsection{Věk}

\includegraphics{graphs/img/vek.png}

\subsubsection{Studuji}

\includegraphics{graphs/img/studuji.png}

\subsubsection{(Jen žáci a studenti) Jsem žák / student}

\subsubsection{Moje nejvyšší dosažené vzdělání}

\includegraphics{graphs/img/moje_nejvyssi_dosazene_vzdelani.png}

\section{Volná otázka \uv{Chci dodat}}

\section{Vypovídací hodnota sebraných dat}

(Ne)Ochrana před vícenásobným vyplněním

Nepravdivé informace

Reprezentativnost vzorku vzhledem k počtu aktivních uživatelů

Možná selekce uživatelů (ti, které vůbec nezajímá redakční obsah,
nemají rádi autora dotazníku apod.)

\section*{Poděkování}

Závěrem je třeba poděkovat všem, kdo se na dotazníku podíleli:
\suser{psycho-kat}, jejíž otázka po \uv{tvrdých datech}
vznik dotazníku vyprovokovala;
\suser{JiKu}, \suser{slu-nicko}, \suser{plihalik}, \suser{mia-maru},
\suser{Papo} a \suser{alweryon}, kteří pročetli otázky před spuštěním
dotazníku a poskytli mi k nim zpětnou vazbu;
šéfredaktorce \suser{Kollenka}, která článek s pozváním k vyplnění
dotazníku na celý týden umístila do bloku redakčního obsahu,
takže se dostal
k mnohem více uživatelům, než by běžný příspěvek z osobního blogu
kdy mohl;
zejména ale patří velký dík každému, kdo věnoval kus svého času
a dotazník vyplnil.

\tableofcontents

\printbibliography

\end{document}
