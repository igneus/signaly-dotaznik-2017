\documentclass[12pt, a4paper, twoside]{article}

\usepackage{fontspec}
\setmainfont[Ligatures={TeX}]{TeXGyrePagella}

\usepackage[czech]{babel}
\usepackage{csquotes}
\usepackage[left=2.5cm, right=2.5cm, top=2.5cm, bottom=2.5cm, marginparsep=3mm]{geometry}

\usepackage[
  backend=biber,
  style=iso-authoryear,
  sortlocale=cs_CZ,
  maxnames=3,
  firstinits=true,
]{biblatex}

\usepackage[show]{ed} % editorial annotations
\usepackage{color}
\usepackage{xcolor}
\usepackage[hidelinks]{hyperref}
\usepackage{changepage}
\usepackage{nameref}
\usepackage{graphicx}
\usepackage{float} % necessary for figures with placement [H]
\usepackage{xspace}
\usepackage{parskip}

\newcommand{\suser}[1]{\href{https://www.signaly.cz/#1}{\texttt{@#1}}}

\newcommand{\answercount}[1]{Odpovědělo \input{graphs/counts/#1.txt} respondentů}

\newcommand{\includegraph}[2]{
  \begin{figure}[H]
    \centering
    \textbf{#2}
    \includegraphics{graphs/img/#1.pdf}
    \answercount{#1}
  \end{figure}
}

\newcommand{\includegraphonly}[2]{
  \begin{figure}[H]
    \centering
    \textbf{#2}
    \includegraphics{graphs/img/#1.pdf}
  \end{figure}
}

\newcommand{\qtype}{\textbf{Typ otázky:}
}
\newcommand{\pickOne}{Výběr jedné možnosti z předdefinované sady\xspace}
\newcommand{\pickMultiple}{Výběr více možností z předdefinované sady\xspace}
\newcommand{\withOther}{s možností formulovat vlastní volnou odpověď (\uv{Jiné})\xspace}
\newcommand{\series}{(Řada otázek stejného typu se stejnými možnostmi odpovědi.)\xspace}
\newcommand{\freeEntry}{Volná odpověď (textové pole)}


\setcounter{secnumdepth}{3}
\setcounter{tocdepth}{3}

\extrafloats{200} % greater limit of floats awaiting placement

\bibliography{biblio}

\author{Jakub Pavlík}
\title{Vyhodnocení dotazníku\\ \uv{signaly.cz z pohledu uživatelů}}

\begin{document}

\setlength{\parindent}{0.5cm}

\maketitle

\section*{Úvod}

Signály letos slaví deset let provozu jako sociální síť.
Kromě toho, že se slaví a vzpomíná, však naléhavě vyvstávají otázky
ohledně dalšího směřování celého projektu:
není třeba provoz na Signálech pozorovat zvlášť dlouho ani pozorně,
aby pozorovatel pochopil, že heslo z titulní stránky
\emph{33 637 mladých křesťanů skrze signály.cz tvoří společenství}
se s realitou hrubě rozchází:
web sice snad má přes 33000 registrovaných
uživatelů, převážná většina uživatelských účtů však je buďto
úplně opuštěná, nebo vykazuje jen minimální aktivitu.
Navíc se ukazuje -- jak bude vidět i dále ve výstupech z dotazníku --
že uživatelskou základnu spíše než mladí křesťané
tvoří \emph{křesťané, kteří v době spuštění Signálů jako komunitního
  webu byli mladí:}
uživatelská základna stárne, Signály nejsou příliš úspěšné
v oslovování nových uživatelů z řad mládeže.
Mezi deklarovaným posláním a tím, kým a k čemu jsou Signály skutečně
využívány, zeje propast, která se nadále rozevírá.

Tyto skutečnosti si žádají nové definování vize a poslání Signálů.
Přispět k tomu jsem se pokusil s pomocí dotazníku, jehož vyhodnocení
překládám. V základech tohoto podniku stojí předpoklad,
který jsem formuloval již v úvodu k dotazníku:
\emph{hlavní hodnotou sociální sítě jsou její uživatelé --
  a cesta k uživatelům novým vede přes pochopení potřeb těch
  stávajících}.

Šetření probíhalo v týdnu 26.~7.--2.~8. 2017
s pomocí dotazníku postaveného v Google Forms.
K oslovení respondentů byl využit článek na osobním blogu
\url{http://dromedar.signaly.cz}, redakcí Signálů pro ten účel
na celý týden laskavě vystavený v bloku redakčního obsahu
na titulní stránce, takže se dostal i k těm, kdo daný blog normálně
nesledují.
Za týden, kdy byl dotazník propagován a probíhal sběr odpovědí,
se jich podařilo nasbírat 215. Je potřeba zohlednit to, že
šetření probíhalo v době dovolených, po poměrně krátkou dobu
a bez masivnější \uv{marketingové podpory}.
I přesto počet sebraných odpovědí jistě má určitou vypovídací
hodnotu ohledně velikosti aktivní části uživatelské základny.

Na následujících stránkách budou nejprve představeny otázky,
na které respondenti odpovídali, spolu s grafy prostého
rozložení odpovědí. Následně nabídneme jejich interpretaci.

\section{Data a zdrojové kódy}

Sebraná data, spolu s kompletními zdrojovými kódy pro sestavení
této publikace (sazba, skripty generující z dat grafy)
a historií jejího vzniku,
jsou na Githubu v repozitáři\\
\url{https://github.com/igneus/signaly-dotaznik-2017}.

\section{Otázky}

Otázky byly rozděleny do šesti bloků.
První blok (\nameref{sec:mojeaktivita})
je nepříliš sourodá skupina otázek bez opravdového společného tématu.

Další tři
(\nameref{sec:kcemu}, \nameref{sec:funkcionality}, \nameref{sec:jinesite})
se z různých stran snaží dobýt odpověď na otázku, čím Signály jsou
pro své uživatele, jako co jsou používány.

Předposlední blok otázek (\nameref{sec:ostatniuzivatele})
reaguje na hlasy, že na Signálech je nevlídné prostředí
a je třeba ho \uv{zútulnit} -- ať už propracovanějším systémem nastavení
soukromí, nebo postihy uživatelů, kteří jsou vnímáni jako problémoví.
Odpovědi na otázky tohoto bloku by měly ukázat, jak velká část
uživatelské základny tyto problémy vnímá jako palčivé.

Otázky závěrečného bloku (\nameref{sec:osobni})
slouží ke strukturování odpovědí
podle vybraných antropologických proměnných.

\vfill

\subsection{Moje aktivita na Signálech}\label{sec:mojeaktivita}

\subsubsection{Na Signálech mám profil}\label{sec:mamprofil}

\qtype \pickOne.

\includegraph{na_signalech_mam_profil}{}

\subsubsection{O Signálech jsem se poprvé dozvěděl/a}

\qtype \pickOne.
Kromě možností zastoupených v grafu byla v nabídce ještě jedna,
kterou nikdo nevyužil: \uv{na Facebooku}.

\includegraph{o_signalech_jsem_se_poprve_dozvedel}{}

\subsubsection{Na Signály chodím}

\qtype \pickOne.

\includegraph{na_signaly_chodim}{}

\subsection{K čemu Signály používám}\label{sec:kcemu}

\subsubsection{Na Signály chodím hlavně}

\qtype \pickMultiple \withOther.

Kromě předdefinovaných odpovědí (seznam níže)
se z volných odpovědí vyprofilovaly tři drobné skupinky
\uv{ŽMM}, \uv{Sleduji diskuse} a \uv{Slovo na den}.

\begin{itemize}
\item Kvůli kontaktu s kamarády z okolí (město, škola, farnost, …)
\item Kvůli kontaktu s kamarády, které moc často nevídám
\item Kvůli kontaktu s kamarády, které znám jenom ze Signálů
\item Protože tu píšu blog
\item Protože rád čtu blogy
\item Protože rád čtu redakční obsah
\item Kvůli přehledu křesťanských akcí
\item Protože rád diskutuji
\end{itemize}

\includegraph{na_signaly_chodim_hlavne}{}

\subsubsection{Signály jsem využil/a k}

\qtype \pickMultiple \withOther.

\begin{itemize}
\item Psaní blogu
\item Čtení blogů
\item Čtení redakčního obsahu
\item Nalezení akce, na kterou jsem pak jel/a
\item Diskutování s ostatními uživateli
\item Kontaktu s kamarády, které znám osobně (z místa bydliště, z DCM apod.)
\item Hledání nových kamarádů
\item Hledání partnera/partnerky
\end{itemize}

\includegraph{signaly_jsem_vyuzil_k}{}

\subsubsection{Na Signálech \uv{se mi povedlo}}\label{sec:uspechy}

\qtype \pickMultiple \withOther.

\begin{itemize}
\item Přečíst si článek, který mě povzbudil
\item Přečíst si článek, který se mi líbil
\item Přečíst si článek, kde jsem se něco nového dozvěděl/a
\item Najít bloggera, kterého od té doby rád/a čtu
\item Napsat článek, který četlo hodně lidí
\item Napsat článek, který se někomu líbil
\item Napsat článek, pod kterým se rozhořela diskuse
\item Obhájit v diskusi svůj názor
\item Díky diskusi opravit svůj mylný názor
\item Díky diskusi si rozšířit obzory
\item Vyvést někoho z omylu
\item Někoho "pořádně setřít"
\item Lépe poznat nauku vlastní církve
\item Lépe poznat nauku jiné církve
\item Být znejistěn ohledně nauky vlastní církve
\item Získat plnomocné odpustky
\item Najít křesťanskou akci, na kterou jsem pak jel
\item Psát si se svými kamarády
\item Najít nové kamarády
\item Najít partnera/partnerku
\end{itemize}

\includegraph{na_signalech_se_mi_povedlo}{}

{\footnotesize
  Možnost \uv{Získat plnomocné odpustky} jsem zařadil jenom
  z rozpustilosti, jako roztomilý nesmysl. Hrstka uživatelů,
  která tuto možnost zvolila, mě však uvrhla v pochybnost:
  je to známka, že si respondenti vtipu všimli a líbil se jim,
  nebo někdo skrze Signály odpustky \emph{opravdu} získal?
  Pokud se vám snad na Signálech opravdu povedlo získat plnomocné
  odpustky, napište mi, jak jste to dokázali!
}

\subsection{Funkcionality, které využívám}\label{sec:funkcionality}

\subsubsection{Jsem autorizovaný/á}\label{sec:autorizovany}

\qtype \pickOne.

\includegraph{jsem_autorizovany}{}

\subsubsection{Jak často využívám jednotlivé funkcionality}

\qtype \pickOne. \series

\includegraph{funkcionality_status_pisu}{Na svou zeď píšu}

\includegraph{funkcionality_status_ctu}{Statusy přátel čtu}

\includegraph{funkcionality_status_komentuji}{Statusy přátel komentuji}

\includegraph{funkcionality_sledovane_blogy_ctu}{Svoje sledované blogy čtu}

\includegraph{funkcionality_blogy_vyber_ctu}{Výběr z blogů čtu}

\includegraph{funkcionality_blogy_nesledovane_ctu}{Blogy, které nesleduji, ani nejsou ve výběru, čtu}

\includegraph{funkcionality_blog_komentuji}{Komentář pod nějaký blog píšu}

\includegraph{funkcionality_blog_pisu}{Příspěvek na blog píšu}

\includegraph{funkcionality_fotky_prohlizim}{Fotky si prohlížím}

\includegraph{funkcionality_fotky_vkladam}{Fotky vkládám}

\includegraph{funkcionality_akce_prohlizim}{Nabídku akcí si prohlížím nebo hledám akci, kam pojedu}

\includegraph{funkcionality_akce_vkladam}{Vkládám akci}

\includegraph{funkcionality_zed_spolecenstvi_ctu}{Zeď společenství čtu}

\includegraph{funkcionality_zed_spolecenstvi_pisu}{Na zeď společenství píšu}

\includegraph{funkcionality_videa_prohlizim}{Na videa se dívám}

\includegraph{funkcionality_videa_vkladam}{Videa vkládám}

\includegraph{funkcionality_autorizovani_chat}{(Vyplňují jen autorizovaní uživatelé) Chat používám}

Pokud na poslední otázku odpověděli i někteří uživatelé,
kteří v \ref{sec:autorizovany} uvedli, že autorizovaní nejsou,
jejich odpovědi nebyly do výsledků zahrnuty.

\subsection{Signály a jiné sociální sítě}\label{sec:jinesite}

\subsubsection{Používáš kromě Signálů i jiné sociální sítě? Jak často navštěvuješ}

\qtype \pickOne. \series

\includegraph{jine_site_Facebook}{Facebook}

\includegraph{jine_site_Twitter}{Twitter}

\includegraph{jine_site_Youtube}{Youtube}

\includegraph{jine_site_Instagram}{Instagram}

\includegraph{jine_site_Googleplus}{Google+}

\includegraph{jine_site_Lidecz}{Lidé.cz}

\pagebreak

\subsubsection{Používáš další sociální sítě?}

\qtype \freeEntry.
Všechny odpovědi měly charakter stručného seznamu jedné nebo více
sociálních sítí (nebo jiných služeb, respondenty za sociální sítě
považovaných). Pro účely sestavení grafu byly odpovědi normalizovány
(sjednocení grafických variant, zestručnění formulací apod.).

\includegraph{pouzivas_dalsi_socialni_site}{}

\subsubsection{Proč používáš Signály, i když je dnes bohatá nabídka jiných sociálních sítí?}\label{sec:konkurence}

\qtype \pickOne. \series

\includegraph{proc_signaly_jenom_signaly}{Používám jenom Signály, jiné sociální sítě nepotřebuji}

\includegraph{proc_signaly_signaly_nejdulezitejsi}{Signály jsou pro mě nejdůležitější sociální síť}

\includegraph{proc_signaly_kamaradi_kteri_jinde_nejsou}{Na Signálech jsem kvůli kamarádům, kteří na jiných sociálních sítích nejsou}

\includegraph{proc_signaly_zajimavi_lide_kteri_jinde_nejsou}{Na Signálech jsem kvůli zajímavým lidem, kteří nejsou na jiných sociálních sítích}

\includegraph{proc_signaly_jedinecna_aktivita}{Na Signálech jsem kvůli něčemu, co se děje jenom tam (např. aktivita nějakého společenství)}

\includegraph{proc_signaly_jedinecne_informace}{Na Signálech jsem kvůli informacím, které jinde nejsou}

\includegraph{proc_signaly_oblibene_blogy}{Na Signálech jsem kvůli blogům, které rád čtu}

\includegraph{proc_signaly_krestanske_prostredi}{Na Signálech jsem proto, že mám rád křesťanské prostředí}

\subsubsection{Pokud jseš na Signálech (ještě) kvůli něčemu jinému, můžeš nám to prozradit tady:}

\qtype \freeEntry.
Odpovědi byly částečně uspořádány podle témat.
\answercount{proc_signaly_jine}.

\begin{itemize}
\item Hlavně jsem chtěl někde založit blog. Křesťanská sociální síť se zdála být vhodným prostředím :)

\item mam tu blog

\item Kvůli psaní blogu.

\item zvláštní směsice zábavy  a poučení v "jazyce", kterému rozumím

\item Určitou roli v tom hraje setrvacnost na signály jsem chodil ještě v době kdy nebyly sociální síť... signály znám cca. 14 let byl jsem i zapojen do redakce starých signalu

\item Setrvačnost, zvyk

\item ze setrvačnosti

\item ze setrvačnosti

\item částečně ze stereotypu - když už jsem se kdysi registrovala a začala psát blog, občas se tam prostě kouknu.

\item z nostalgie:)

\item Z nostalgie.

\item z nostalgie

\item z nostalgie, z věrnosti

\item Připadají mi mnohem smysluplnější než fb a jiné sociální sítě...

\item žmm

\item kvůli společenství ŽMM

\item Kvůli společenství ŽMM, jinak je tu mrtvo a nechodila bych sem

\item dialog (nejen) s mladou generací

\item Povzbuzují mě vybraná videa

\item Je to prostor pro diskusi mezi křesťany, jaký na českém internetu nemá konkurenci.

\item Na signálech jsem proto, že mě přibližují Bohu a pomáhají mi lépe žít křesťanským životem.

\item nacházím zde inspiraci, jak obohatit farnost, ve které žiji

\item Se zájmem sleduji diskuse různých uživatelů a z nějakého důvodu nejraději právě těch, kteří jsou některými vnímáni jako "šťouralové", přestože ne vždy s nimi souhlasím. Líbí se mi, že dokážou svými příspěvky aspoň některé (například mě) přivést k zamyšlení, někdy k rozšíření obzorů, někdy k přehodnocení stanovisek a někdy třeba jen k utvrzení toho co si už myslím. Sama se diskusí aktivně zúčastňuji jen zřídka, poslední dobou spíše jen lajknutím komentáře, který se mi zdá trefný.

\item Obsah, který zveřejním, vidí pouze věřící známí. Na fb ho vidí i lidé z nevěřícího prostředí, kterým by mohlo sem tam něco připadat dost divné. A ani mě by nebylo příjemné, aby si moje "nábožné" statusy četli.

\item Signály jsou více komornějšía  rodinější, než jiné sociální sítě. Používám je i třeba ke zpětné reakci či vzpomínání na proběhlé akce. Ke komunikaci s uživateli, které mohu potkat osobně.

\item Kvůli maminkovské starosti i radosti

\item Na signálech jsem kvůli křesťanskému prostředí které na jiných soc. síť. není

\item Pro mě jsou signály.cz bezpečnou soc. sítí: bez nepříjemných až sexuálně motivovaných reklam (sponzorované příspěvky na FB), autorizovaní uživatelé (skvělá prevence anonymity).

\item pro povzbuzení

\item práce

\item inspirace k duchovnímu růstu. taky tady hledám vtipy vhodné do farního časopisu, recenze knížek.

\item Vkládám jsem fotky

\item dřív to bylo kvůli společenství křesťanských uživatelů, četla jsem povzbuzující a zajímavé články od přátel, jezdila jsem na akce, kde signály byly a potkávala lidi z týmu... dnes veškerá aktivita klesla, možná je to tím, že lidi, kteří byli aktivní v té době už teď mají své rodiny a jsou aktivnější v reálu a mladší kamarádi jsou na facebooku

\item Signály jsou pro mě určitá náhrada facebooku. Znám se a vím, že kdybych měla facebook, tak jsem na něm hodně často a šťourala bych se ve věcech, který by mi nebyly k prospěchu. Signály jsou v tomto bezpečnější, protože tady není tak velká aktivita, takže nezabiju tolik času - za 15 minut mám po prohlížení :D A navíc sem lidi nedávají věci, který bych o nich radši nevěděla, jak je tomu jinde.

\item jen k tomu posledním - křesťanské prostředí..  trochu:-). Také ta starší podoba signálu byla myslím lepší.("Modernější" není v každém případě lepší)

\item je to křesťanstská sociální síť
\end{itemize}

\subsection{Já a ostatní uživatelé}\label{sec:ostatniuzivatele}

\qtype \pickOne. \series

\includegraph{ostatni_tesi_me_libi_se_kamaradi}{Těší mě, když se mým kamarádům líbí můj obsah (status, fotka, blog, video)}

\includegraph{ostatni_tesi_me_libi_se_cizi}{Těší mě, když se cizím lidem líbí můj obsah}

\includegraph{ostatni_tesi_me_komentar_kamaradi}{Těší mě, když moji kamarádi komentují můj obsah}

\includegraph{ostatni_tesi_me_komentar_cizi}{Těší mě, když cizí lidé komentují můj obsah}

\includegraph{ostatni_tesi_me_zprava_kamaradi}{Těší mě, když mi kamarádi píšou soukromé zprávy}

\includegraph{ostatni_tesi_me_zprava_cizi}{Těší mě, když mi cizí lidé píšou soukromé zprávy}

\includegraph{ostatni_autorizovani_duveryhodnejsi}{Autorizovaní uživatelé jsou pro mě důvěryhodnější než neautorizovaní}

\includegraph{ostatni_neprijemne_komunikace_obtezuje}{Komunikace na Signálech mě někdy obtěžuje, nebo je mi z ní těžko}

\includegraph{ostatni_neprijemne_neprijemne_situace}{Na Signálech jsem zažil/a situace, které jsou mi nepříjemné}

\includegraph{ostatni_neprijemne_zazivat_nechci}{Na Signálech jsem zažil/a situace, které zažívat nechci}

\subsubsection{Na Signálech jsem zažil/a}\label{sec:neprijemneco}

\qtype \pickMultiple.

\includegraph{neprijemne_co}{}

\subsubsection{(Jen ti, kdo zažili nepříjemné komentáře) Komentáře mi byly nepříjemné, protože}

\qtype \pickMultiple \withOther.

Do výsledků této a následující otázky byly zahrnuty
jen odpovědi respondentů,
kteří v \ref{sec:neprijemneco} uvedli, že zažili nepříjemné komentáře
od kamarádů nebo od cizích lidí.

\begin{itemize}
\item Byly od kamaráda
\item Byly od cizího člověka
\item Byly od někoho, kdo je o hodně mladší
\item Byly od někoho, kdo je o hodně starší
\item Nevěděl/a jsem, od koho jsou
\item Vyjadřovaly jiný názor, než mám já
\item Pisatel psal, že nemám pravdu
\item Dělaly si ze mě legraci
\item Urážely mě
\end{itemize}

\includegraph{neprijemne_komentare_protoze}{}

\subsubsection{(Jen ti, kdo zažili nepříjemné komentáře)}

\qtype \pickOne. \series

\includegraph{neprijemne_komentare_zvazoval_odchod}{Uvažoval/a jsem, že kvůli nepříjemným komentářům ze Signálů odejdu}

\includegraph{neprijemne_komentare_zakazat}{Na Signálech by mělo být zakázané psát takové komentáře}

\includegraph{neprijemne_komentare_pisatele_pryc}{Ti, kdo píší takové komentáře, by na Signálech vůbec neměli být}

\subsubsection{(Jen ti, kdo zažili nepříjemné vzkazy) Vzkazy mi byly nepříjemné, protože}

\qtype \pickMultiple \withOther.

Do výsledků této a následující otázky byly zahrnuty
jen odpovědi respondentů,
kteří v \ref{sec:neprijemneco} uvedli, že zažili nepříjemné vzkazy
od kamarádů nebo od cizích lidí.

\begin{itemize}
\item Byly od kamaráda
\item Byly od cizího člověka
\item Byly od někoho, kdo je o hodně mladší
\item Byly od někoho, kdo je o hodně starší
\item Nevěděl/a jsem, od koho jsou
\item Vyjadřovaly jiný názor, než mám já
\item Pisatel psal, že nemám pravdu
\item Dělaly si ze mě legraci
\item Urážely mě
\end{itemize}

\includegraph{neprijemne_vzkazy_protoze}{}

\subsubsection{(Jen ti, kdo zažili nepříjemné vzkazy)}

\qtype \pickOne. \series

\includegraph{neprijemne_vzkazy_zvazoval_odchod}{Uvažoval/a jsem, že kvůli nepříjemným vzkazům ze Signálů odejdu}

\includegraph{neprijemne_vzkazy_zakazat}{Na Signálech by mělo být zakázané psát takové vzkazy}

\includegraph{neprijemne_vzkazy_pisatele_pryc}{Ti, kdo píší takové vzkazy, by na Signálech vůbec neměli být}

\pagebreak % ZLOM

\subsection{Osobní údaje}\label{sec:osobni}

\subsubsection{Jsem}\label{sec:pohlavi}

\qtype \pickOne.

\includegraph{jsem}{}

\subsubsection{Věk}\label{sec:vek}

\qtype \pickOne.
Nad možnosti zachycené v grafu byly nabízeny také nikým nevyužité
\uv{méně než 10 let} a \uv{11-12}.

\includegraph{vek}{}

\subsubsection{Studuji}

\qtype \pickOne.

\begin{itemize}
\item ano (jsem žák ZŠ nebo student SŠ, VOŠ, VŠ)
\item ne, vzdělání mám již ukončené
\end{itemize}

\includegraph{studuji}{}

\subsubsection{(Jen žáci a studenti) Jsem žák / student}

\qtype \pickOne \withOther.
Volné odpovědi, které bezpečně zapadaly do jedné z předdefinovaných
kategorií, byly normalizovány.

\begin{itemize}
\item ZŠ
\item SŠ - učňovský obor
\item SŠ - učňovský obor s maturitou
\item SŠ - všeobecná (gymnázium nebo obdobný typ školy)
\item SŠ - s odborným zaměřením (průmyslová, obchodní apod.)
\item VOŠ
\item VŠ - humanitní obor
\item VŠ - přírodovědný obor
\item VŠ - technický obor
\end{itemize}

\includegraph{studuji_co}{}

\subsubsection{Moje nejvyšší dosažené vzdělání}

\qtype \pickOne \withOther.
Volné odpovědi, které bezpečně zapadaly do jedné z předdefinovaných
kategorií, byly normalizovány.

\begin{itemize}
\item ZŠ
\item SŠ - učňovský obor
\item SŠ - učňovský obor s maturitou
\item SŠ - všeobecná (gymnázium nebo obdobný typ školy)
\item SŠ - s odborným zaměřením (průmyslová, obchodní apod.)
\item VOŠ
\item VŠ - humanitní obor
\item VŠ - přírodovědný obor
\item VŠ - technický obor
\end{itemize}

\includegraph{moje_nejvyssi_dosazene_vzdelani}{}

{\footnotesize
Ješitnost autorovi dotazníku velí zdůraznit, že volná
odpověď \uv{věčného studenta} \emph{nepochází} od něj.
Ač také věčný student, nemá potřebu schovávat se před
trapnou skutečností, že jeho nejvyšší \emph{dosažené} vzdělání
je jen středoškolské.
}

\subsection{Chci dodat}

\qtype \freeEntry.
Odpovědi byly roztříděny podle témat. (Obsáhlejší odpovědi
dotýkající se více témat byly vždy ponechány vcelku
a zařazeny pod to téma, kam se zdály nejvíce příslušet.
Pokud snad něčí odpověď byla naporcována, stalo se to redakčním
omylem, ne záměrně.)

\subsubsection{Náměty na rozvoj signaly.cz}

\setlength{\parskip}{0.3cm}

Uvítala bych vyhledávání dle určitého slova ve společenství na nástěnce (vrácení se k některým příspěvkům)

Vrátil bych signaly.cz do podoby z roku 2008

stránky by se mohly sami aktualizovat

myslím, že by bylo super udělat tu například nějakou miniseznamku nebo nějaké společenství, kde by se lidé mohli seznamovat :)

Mohla by být užitečná otázka, ohledně pomoci, rady od vrstevníka, laika, kněze

"Bylo by sympatické, kdyby signály byly ""trychtýřem"" pro další informační zdroje - veřejné blogy zajímavých osobností i zahraničních, lidí jako Orko Vácha i jiných (ne nutně katolíků), připadají mi v tomto směru prostě spící. Člověk se díky facebooku dozví spoustu informací a bylo by fajn překlopit tuto situaci i do křesťanského prostředí. Možná už se o to v současnosti snažíte, ale buď je těch informačních/ zajímavostních impulzů málo nebo jsou interpretovány nějak špatně, protože mám dojem, že se ke mě nedostávají. Jako je ""výběr z blogů"" líbil by se mi i nějaký ""výběr z venku"".
 A vůbec bych se nezlobila, kdyby se signály trochu převlékly. Za ty roky se ta zelená docela okouká... :(
Držím palce +"

Asi bych netlačila tolik na to, že signály jsou pro mladé. Nemůže to být prostě křesťanský komunitní web? A pak mi přijde líto, že na signálech je spousta hodnotného materiálu, ale schované někde v bažinách společenství. Např. čteme z youcat. Nešlo by tu trochu uklidit a věci dohledatelně porovnat? Díky

Podle mě by nemělo být cílem maximálně rozšiřovat komunitu. Raději umožnit nějak systematicky (podle štítků) roztřídit články; je škoda, když starší články zapadnou a propagují se jen nové.

"Dotazník je úžasný. Jednoduchý, stručnýa  přehledný.
Nebylo by špatné popřemýšlet o tom, aby dnešní signály navázaly na to, kvůli čemu vznikly a na dobu, kdy signály byli skutečně živé.
Některé prvky, které dnes zmizeli by nebylo špatné obnovit (diskuzní společenství; více motivovat mladé k psaní článků na blogy; propojit virtuální signály s realitou - být více vidět na různých akcích, nejen na těch velkých, ale i menších; nabídnout nové možnosti pro starší, kteří odrostli a mají rodiny; zkusit z části navrátit grafiku, aby signály nebyli tolik podobné FB)."

toužím po informovanosti a nebát se jiných názorů, kéž mě signály podpoří v komunikaci či sledování komunikace o závažných tématech co se křesťanů týkají. netrkat hlavu do písku a neplytvat energií

Myslím si, aby měly signály možnost oslovit stávající mladé křesťany, musí se to dít ve spolupráci s faráři, nebo alespoň kaplany pro mládež, které je nutno namotivovat tak, aby si akci vzali za své - taktéž biskupové, od nichž by měl popud vyjít směrem k farářům - a také s řády pracujícími s mládeží (Salesiáni, apod.), investovat do propagace ve farnostech a na školách - např. skrz vytvoření výukového programu o náboženství, filozofii, humanitních vědách, apod., který bude přijatelný pro školy - edukativní složka, která přiměje mladé (i ty nevěřící hledající) na signály přijít a případně se zapojit i jinak. Což lze zase ve spolupráci s katechety, či studenty vysokých škol těchto oborů. Jsem tedy velký podporovatel e-learningu (humanitní vědy, náboženství, osvěta, apod.), zajištěné výlepy plakátů do měst (nejen farností, ale samozřejmě ve spolupráci s farností a dohodnutých fin. podmínek), nebo se zamyslet i nad reklamou v novinách - je to vyšší částka, ale okolo církevních svátků, apod. si lidé kladou různé otázky - udělat si marketingovou strategii a říct si, jestli chceme evangelizovat, nebo jen chytat v našich vodách. Obávám se, že ti akční jedinci, kterých je nedostatek na signálech, jsou právě tak akční, že už nic jiného nestíhají a nevidí ve webu nic hmatatelného. Vidím to sama na sobě, nemám u práce a jiných bohulibých aktivit na nic dalšího čas, tak i můj blog skomírá prázdnotou, ale i na druhou stranu se potýkám s tím, že kdybych chtěla napsat i něco jiného, co má sice říz, ale není to vyloženě křesťanského tématu, tak by to stejně nikoho nezajímalo, protože se na signálech nepsaným pravidlem očekává, že téma bude zbožné, tudíž na "nezbožné" nebo ne primárně zbožné úvahy bych si stejně musela hledat jinou komunitu, nebo mi to za tu tvorbu ani v únavě všedního dne nakonec nestojí - nejsem motivována. Atd. Sur sum corda!

\subsubsection{Výrazy náklonnosti k signaly.cz}

Signály potřebuji a děkuji všem za jejich práci pro nás.

Signály jsou super.

Signály jsou skvělá věc.

Signály jsou nejlepší sociální síť!

Signály jsou bezpečný prostor pro lidi věřící. Věřím, že tady ještě dlouho budou. Díky za ně.

Signálům fandím. Myslím, že i přes všechny problémy má tenhle projekt v nějaké podobě smysl. (pro starší i mladší)

jsem rada,ze jsou...

Jsem ráda, že Signály existují. Snad si na CSM najdu přátelé, se kterými budu přes Signály zažívat jen hezké situace a plánování akcí. ☺

Děkuji, že signály jsou a fungují :)

Děkuji za signály

Doufám, že Signály přetrvají, je tu třeba mít křesťanskou sociální sit...

díky za signály ;-)

Díky Bohu za Signály, jsem velmi potěšená, že mohu chodit na křesťanské stránky, velmi si jich vážím a fandím jim! Díky všem, kteří se na nich podílejí, Pán Bůh zaplať všem!

Signaly jsou skvely, doufam ze se jim bude darit i nadale..:-) Ze Realizacni tym udela signaly.cz jeste atraktivnejsi, aby pomohli treba nezadanym krestanum z ruznych denominaci..

Děkuji za dotazník. Ráda bych, kdyby signály zůstaly víceméně takové jaké jsou. V případě, že se zavede něco jako všeobecná cenzura "nepříjmných komentářů a názorů" asi bych ztratila zájem je více využívat, protože se domnívám že by tímto "sítem" propadla většina mých oblíbených diskutujících, což by byla podle mě škoda.

\subsubsection{K otázce poslání signaly.cz}

Velmi se mi libi pestrost lidi, starsi intelektualove a konzervativci, mlade poeticke zeny, jeste "svezi" mladeznici, je to nejen krestansky unikatni, ale celkove unikatni a diky tomu hodnotna socialni sit, ktera by nemela chybet!

Signály jsou pro mě ze všeho nejvíc blogová platforma. Popravdě, místní společenství mi vůbec nic neříkají a ta, který by snad mohla, jsou mrtvá.

signály jsou podle mě ideální prostor pro svědectví o osobním vztahu s Bohem, o Boží lásce, to je podle mě hlavní přidaná hodnota oproti jiným sociálním sítím

Mám dojem, že na rozdíl od např. FB signály lépe umožňují diskuse lidí mimo vlastní sociální bublinu (kromě křesťanství, ale to je takový vstupní předpoklad...)

ŽMM má obrovský potenciál, matky na mateřské jsou nejaktivnější na internetu. Matka a manželka sice ještě nejsem, ale i tak je to pro mě velmi obohacující a zajímavé....

\subsubsection{Proč už pro mě osobně signaly.cz nejsou tak hodnotné jako dřív}

Změna Signálů podle vzoru Facebooku byla chyba. Mnohem větší chybou však byla perzekuce a vyštvání tradicionalistů a jiných "potížistů".
z mladších generací je tu málo uživatelů

signály.cz pro mě měly velkou důležitost před 5-10 lety, kdy jsem jako mladá pořádala akce a signály nabízely dobrou možnost propagace a také poskytovaly velkou nabídku dalších akcí, kterých jsme se se spolčem účastnili. Také signály.cz sloužily jako dobrý komunikační prostor pro spolčo (vlastní skupina) a prostor pro komunikaci s přáteli. Dnes se mí přátelé přesunuli ve velké míře na facebook a signály v tomto směru ztratily svou atraktivitu. A mladí z mého okolí nemají o signálech ani tušení. Já vám chci vyjádřit osobní díky za vaši práci při zdokonalování webu :)

Na dřívější podobě signálu měl člověk trochu "domovský pocit" v této podobě se to nějak ztrácí. Ale je tam i dost dobrých věcí, takže celkově - je dobré, že jsou.

Jedním z mála důvodů, proč jsem občas na Signály zavítala, byl blog Toba (Karla Skočovského). Kam zmizel? Věčná škoda všech jeho článků, které si už nikdo nepřečte!

"Chtěl bych dodat, že podle mého názoru se signály postupem času až příliš připodobnily facebooku, se kterým však nemohou konkurovat. Proto se samy postavily do role ""slabšího konkurenta"". Drtivá většina lidí na signálech facebook užívá, takže je si toho vědoma rovněž. Uvítal bych, kdyby signály udělaly trochu ""odklon"" od mainstreamu sociálních sítí a šly více vlastní cestou.
I já osobně jsem mnohem častěji a radši chodil na signály, když vypadaly svébytně a nebyly křesťanským ""klonem"" facebooku. Od doby, kdy proběhly aktualizace a změna vzhledu jsem u sebe vypozoroval, že na signály chodím čím dál méně často a čím dál méně je používám. Rovněž méně často píši na blog. Úplně mi vystačí facebook kde stejně všichni známí jsou."

\subsubsection{K "nepříjemným zkušenostem" na signaly.cz}

"Na signálech (ale i obecně v životě) mi chybí větší důraz na pravidla komunikace. Ta virtuální má svá specifika (tvrdí se, že cca 80% komunikace probíhá na neverbální bázi, která je přes síť nezachytitelná) a mým názorem je, že 99% ""nepříjemných pocitů"" vzniká právě z neznalosti těchto specifik.
Druhou otázkou je ""církevní nauka"". Myslím, že zde často dochází ke zmatení. Ale naprosto netuším, jak by se to dalo ""korigovat"". Snad asi vedením uživatelů k tomu, aby si věci, které jsou pro ně důležité, ověřovali v jiných pramenech (jakých, kde apod.)"

"Chci poděkovat za mnohahodinovou práci věnovanou tomuto dotazníku. Vážím si toho. Signály jsou totiž mojí jedinou sociální sítí, kde komunikuju s lidma.
K situaci kolem komentářů: ano, zažila jsem nepříjemné situace, chtěla jsem ze Signálů odejít. Dala jsem si malou pauzu a pak využila možností, které jako uživatel mám - zavřít komentáře pod mými články (a tím se zbavit i pozitivní zpětné vazby), nečíst košaté diskuze, které mě přestaly obohacovat... A ejhle - pozitivní zpětné vazby začaly přicházet spontánně formou soukromých zpráv.
Všimla jsem si, že se občas na Signálech vyskytují anonymní profily, jejichž cílem je vyloženě provokovat, urážet. Dokonce rýpavým způsobem shazovat autoritu papeže. Myslím si, že by s tím realizační tým měl něco dělat. Pokud budu mít čas, zkusím dohledat konkrétní příklad. "

\subsubsection{Reakce neregistrovaných uživatelů (!)}

Nejsem na Signálech registrovaná a ani to neplánuju, ale "chodím" na ně celkem často a záleží mi na nich.

signaly.cz sleduji minimálně už 6 let, poslední 3 roky intenzivněji (denně), dosud jsem si ale nezaložila vlastní profil

\subsubsection{Věcná zpětná vazba k dotazníku}

tenhle dotazník je dost dlouhej

Dobry napad, libi se mi i provedeni, nezaznamenala jsem sugestivni otazky :)

Děkuji všem, kteří práci na Signálech věnují svůj čas!
část dotazníku, kde se vyplňuje, co vše jsem na signály.cz zažil (jel jsem na akci atd.) není moc relevantní, neboť není časově vymezená např. v posledním roce -protože mnohé zažité věci jsou (jistě nejen u mě) z dřívější aktivnější éry signály.cz.

\subsubsection{Upřesnění k odpovědím na strukturované otázky}

Ohledne těch nepříjemných vzkazu/komentaru berte to s rezervou je to tak šest let zpátky.

Kdyz mam po maturite, tak nevim, co mam dat, ze studuji. Ale to je ted u vsech dotazniku :D.

\subsubsection{Pozdravy, díky a jiné méně věcné odpovědi}

velmi pěkný dotazník těším se na vyhodnocení

těším se na výsledky. díky

Tento dotazník byl výborný nápad! Díky.

S Boží pomocí, je to na Něm, jak to bude!

přeji pěkný den :-)

Jen pokračujte v Božím díle +

Good job!

Díky =)

Děkuju za dotazník :)

Díky za práci se sestavením dotazníku.

Díky za dotazník, těším se na vyhodnocení.

Díky za vaše snahy o růst a zvyšování kvality Signálů...

Díky za tento dotazník, snad přinese své plody!

Dotazník je super, dík že sis s ním dal tu práci. Doufám, že z něho vzejde něco zajímavého. Anonym

Hotovo, snad Vám to nějak pomůže. Děkuji za příjemný dotazník. Vyprošuji Vám Boží požehnání +

Díky borci, který má zájem o signály a jejich budoucnost za vytvoření dotazníku ;) jsem zvědav, co se bude dít. Veni Sancte Spiritus :)

\subsubsection{Jiné}

Signály.cz by měly rozjet marketing, který nakopne uživatele k jejich většímu používání.

signály jako takové mi přijdou super. úplně jde cítit, že nejsou facebookem apod, je tam úplně jiné prostředí, jiný duch, lidé tam nereagují tak jako reagují lidé ve skupinách na fb, kterých jsem kdysi byla součástí. nemám nic proti signálům, jako takové je velmi podporuji a říkám o nich ostatním a zvu je atd, zvláště ty hledající křestanské prostředí, kterého se jim nedostává ve farnosti, mezi ostatními maminkami na mateřské z okolí atp.... ale není to místo pro mě. mám pocit, jako kdyby mě Pán chtěl jinde. nechce, abych se upínala k nějakému společenství, lidem, názorem... chce abych se upnula jen k Němu. někdy mi je tak strašně těžko na duši, když je používám... tedy když to shrnu, signály jsou dle mě super, nápad, vedení i uživatelé, ale nejsou na mé životní či duchovní cestě či jak to nazvat.

signály byly dříve fajn, teď mi to přijde jako mrtvej projekt... ale díky moc lidem, kteří jej zachraňují, myslím však, že obsah by měly především tvořit uživatelé

Bohužel jsem zatím nepřišel na nějaké větší výhody signálů. S kamarády máme názor, že signály nemají větší potenciál.

\setlength{\parskip}{0cm}



\section{Vypovídací hodnota sebraných dat}

Než přikročíme k interpretaci sebraných dat, je třeba věnovat
pozornost skutečnostem, které mají vliv na jejich vypovídací
hodnotu.

\subsection{Pravdivost údajů}

Jako u většiny anonymních internetových dotazníků,
nemáme žádnou jistotu, že respondenti skutečně patří do cílové
skupiny. Dotazník mohl vyplnit každý, kdo se o něm dozvěděl.

Nebylo vyvinuto žádné úsilí ve snaze zajistit, aby každý
respondent mohl dotazník vyplnit právě jednou.
Pokud by to tedy někoho bavilo, mohl dotazníků vyplnit
neomezené množství.

Z povahy dotazníku plyne nemožnost ověřit, zda respondenti
uváděli pravdivé informace.

\subsection{Reprezentativnost vzorku}

I pokud by všichni respondenti byli skutečně uživateli Signálů,
vyplnili dotazník každý právě jednou a pravdivě,
je nezbytné ptát se, nakolik je sebraný vzorek reprezentativní.
Jak velkou část aktivních uživatelů se podařilo podchytit?
Jsou ve vzorku zastoupeny pokud možno všechny specifické
podskupiny, nebo došlo k nějaké selekci?

Dotazník cílil především na aktivní uživatele, tedy na ty,
kdo Signály navštěvují denně nebo častěji. Tomu byla uzpůsobená
doba sběru dat, omezená na jeden týden,
a způsob oslovování respondentů, omezující se na
decentní propagaci na titulní stránce.
Pokud je nějaká významnější skupina uživatelů, která na Signály
chodí pravidelně, ale jen týdně nebo řídčeji, je pravděpodobné,
že ve vzorku není příliš zastoupena.

Šetření probíhalo na přelomu července a srpna, tedy v době
dovolených, kdy lze na sociálních sítích předpokládat spíše
nižší aktivitu uživatelů.
Lze předpokládat, že to mělo vliv na \emph{velikost} vzorku,
který by v jiném ročním období snad byl o něco větší.
Výrazný vliv na jeho \emph{složení} nepředpokládám.

K určité selekci respondentů mohlo dojít vzhledem k formě propagace.
Dotazník byl totiž propagován hlavně článkem zařazeným v bloku
redakčního obsahu na titulní stránce. Redakční obsah je sice
na prominentním místě, všem na očích, ale považuji za pravděpodobné,
že mu i značná část uživatelů věnuje pozornost malou až žádnou.

Dalším možným faktorem selekce je osoba průzkum provádějící:
od začátku totiž šlo o soukromý počin jednoho uživatele.
Autor se pak stěží
může považovat za osobu všeobecně oblíbenou.
Je tudíž možné, že jsou ve vzorku nedostatečně zastoupeni
\uv{ti, kdo nemají rádi \suser{dromedar}a} (a tudíž nemají chuť
zapojovat se do jeho podniků).
Není ale důvod myslet si, že by tato skupina, pokud vůbec existuje,
byla nějak velká nebo charakteristicky profilovaná a její případné
nezapojení se mělo zásadnější vliv na celkový obraz.

\subsection{Předpoklad spolehlivosti}

Viděli jsme, že spolehlivost sebraných dat ohrožuje celá řada
faktorů. Na druhou stranu se ale můžeme ptát:
je nějaký důvod domnívat se, že data v některém ohledu nespolehlivá
skutečně jsou?

Nevšiml jsem si ničeho, co by naznačovalo, že někdo dotazník
záměrně vyplnil opakovaně nebo ho nakrmil nesmyslnými daty.
Nevidím žádný důvod předpokládat principielní nepravdivost sebraných
dat.

Pokud jde o reprezentativnost vzorku, je namístě větší opatrnost.
Pro zpřesnění pohledu by bylo užitečné mít informace
o návštěvnosti webu ve sledovaném období a o skladbě (např. věkové)
aktivních uživatelů, jak je mají k dispozici
správci webu.\footnote{
  Nepamatuji si ovšem, že by data tohoto druhu za celou historii
  Signálů kdy byla zveřejněna, a předpokládám, že se z toho
  spíš nebude dělat výjimka.
}
I zde však snad můžeme předpokládat, že vzorek víceméně
reprezentativní je.

\textbf{Dále tedy budeme předpokládat, že sebraná data jsou v zásadě
pravdivá a představují reprezentativní vzorek uživatelské základny
Signálů.}


\section{Signály jako produkt}

\subsection{Kdo Signály používá}

Z grafů odpovědí na jednotlivé otázky je zřejmé,
že na Signálech výrazně převažují ženy nad muži (\ref{sec:pohlavi})
a nejsilnější věkovou skupinou (\ref{sec:vek}) je skupina
21-40 let. Typickým uživatelem je žena v produktivním věku.
To lze ilustrovat grafem zastoupení pohlaví v jednotlivých
věkových skupinách:

\includegraphonly{vek_x_pohlavi}{Věkové skupiny podle pohlaví}

Většina aktivních uživatelů Signály používá
(nebo tu přinejmenším má profil) opravdu dlouho (\ref{sec:mamprofil}).
Šest nebo více let jsou registrovány více než tři čtvrtiny uživatelů,
více než jedna třetina pak je registrována po celých deset let,
kdy Signály jako sociální síť fungují. Nebude tedy velkým
přeháněním, když řekneme, že Signály -- přinejmenším pokud
jde o uživatelskou základnu -- dodnes žijí ze vkladu prvních let
provozu.

Při pohledu zvenku se může zdát, že aktivní společenství
\uv{Ženy, matky, manželky}\footnote{
  \url{https://www.signaly.cz/zeny-matky-manzelky}},
cílící na ženy v produktivním věku, na Signály přitáhlo
velké množství nových uživatelek a početní převahu této skupiny
dále posílilo. Sebraná data ale tuto domněnku nepotvrzují.
Naopak se zdá, že v poslední době na Signálech žen nijak
dramaticky nepřibývá a ŽMM tedy zřejmě bere členky především
z řad stávajících uživatelek.
(Popř. se zástupy čerstvých \uv{uživatelek-matek-manželek} do výsledků
průzkumu z nějakého důvodu nepromítly, což by ovšem kladlo
velký otazník před náš předpoklad reprezentativnosti sebraných dat.)

\includegraphonly{profil_x_pohlavi}{Stáří uživatelského profilu a pohlaví}

Další nápadnou charakteristikou uživatelů Signálů je poměrně
vysoká míra vzdělanosti. Více než polovinu uživatelů,
kteří mají ukončené vzdělání, tvoří absolventi vysokých
škol.
Podíl vysokoškoláků mezi uživateli Signálů je tak několikanásobně
větší než v celku české populace, kde vysokoškoláci tvoří jen 12,5~\%,
v ročnících čerstvých absolventů pak až ke 20~\%.\footnote{
  Srovnáváno s údaji z výsledků sčítání lidu 2011
  \cite{sl2011vzdelani}
}

\includegraphonly{vzdelani_sloucene}{Dosažené vzdělání -- hrubý pohled}

\subsection{Používané funkcionality}

Srovnání odpovědí na \uv{hodnotové} otázky (\ref{sec:kcemu})
s odpověďmi na otázky po frekvenci využívání funkcionalit
(\ref{sec:funkcionality}) ukazuje, že to, že je nějaká služba
pro uživatele důležitá, neznamená, že ji používá často.
Více než polovina respondentů uvedla, že
na Signály chodí hlavně proto, že rádi čtou blogy,
a podobně i v přehledu \uv{uživatelských úspěchů}
jasně dominují \uv{úspěchy} čtenářské.
Signály se tím jasně profilují jako \textbf{blogovací platforma}.
U otázek na frekvenci čtení blogů však převládají odpovědi
\uv{alespoň jednou týdně} a \uv{méně než jednou týdně}.
Každodenních čtenářů je hrstka.

Po čtení blogů uživatelé používají Signály hlavně
ke \uv{kontaktu s kamarády}, tedy jako \textbf{sociální síť}.
Mezi třemi nabízenými skupinami kamarádů převládají ti,
které uživatelé znají právě jen ze Signálů.

Dalšími pro uživatele důležitými funkcionalitami jsou přehled
křesťanských akcí, možnost diskutovat (od nasazení verze JP2
až donedávan odkázaná na komentáře pod blogy a příspěvky na zdech
uživatelů a společenství) a redakční obsah.

Výše načrtnutý obraz potvrzují i odpovědi na otázku
po \uv{uživatelských úspěších} (\ref{sec:uspechy}).
Dominují úspěchy čtenářské a autorské (vlastní Signálům jakožto
\textbf{blogovací platformě}),
následují úspěchy vlastní \textbf{sociální síti}
a další.

\subsection{Signály v konkurenci}

Z ostatních sociálních sítí uživatelé Signálů nejčastěji používají --
nepřekvapivě -- Facebook (\ref{sec:jinesite}).

Srovnání, jak často používají Facebook aktivní a méně aktivní
uživatelé Signálů, ukazuje, že mezi aktivitou na Facebooku
a na Signálech není žádný vztah. Používanost Facebooku je prakticky
totožná mezi těmi, kdo na Signály chodí denně,
jako mezi těmi, kdo sem zavítají méně často.

\includegraphonly{jine_site_Facebook}{Jak často navštěvuješ Facebook -- všichni}

\includegraphonly{nejaktivnejsi_signalnici_a_fb}{Jak často navštěvuješ Facebook -- ti, kdo chodí na Signály denně}

\includegraphonly{mene_aktivni_signalnici_a_fb}{Jak často navštěvuješ Facebook -- ti, kdo chodí na Signály méně často}

Odpovědi na otázky, které zjišťují hodnotu Signálů v konkurenci
dalších sociálních sítí (\ref{sec:konkurence}),
ukazují, že silou Signálů jsou zejména obsah a aktivity.
Zřejmě není až tak důležité, \emph{kdo} na Signálech je,
ale \emph{co} tu tvoří, resp. co \emph{se tu děje}.


\section{\uv{Sociální ovzduší}}

\section{Sondy}

\section*{Poděkování}

Závěrem je třeba poděkovat všem, kdo se na dotazníku podíleli:
\suser{psycho-kat}, jejíž otázka po \uv{tvrdých datech}
vznik dotazníku vyprovokovala;
\suser{JiKu}, \suser{slu-nicko}, \suser{plihalik}, \suser{mia-maru},
\suser{Papo} a \suser{alweryon}, kteří pročetli otázky před spuštěním
dotazníku a poskytli mi k nim zpětnou vazbu;
šéfredaktorce \suser{Kollenka}, která článek s pozváním k vyplnění
dotazníku na celý týden umístila do bloku redakčního obsahu,
takže se dostal
k mnohem více uživatelům, než by běžný příspěvek z osobního blogu
kdy mohl;
zejména ale patří velký dík každému, kdo věnoval kus svého času
a dotazník vyplnil.

\tableofcontents

\printbibliography

\end{document}
