\documentclass[12pt, a4paper, twoside]{article}

\usepackage{fontspec}
\setmainfont[Ligatures={TeX}]{TeXGyrePagella}

\usepackage[czech]{babel}
\usepackage{csquotes}
\usepackage[left=2.5cm, right=2.5cm, top=2.5cm, bottom=2.5cm, marginparsep=3mm]{geometry}

\usepackage[
  backend=biber,
  style=iso-authoryear,
  sortlocale=cs_CZ,
  maxnames=3,
  firstinits=true,
]{biblatex}

\usepackage[show]{ed} % editorial annotations
\usepackage{color}
\usepackage{xcolor}
\usepackage[hidelinks]{hyperref}
\usepackage{changepage}
\usepackage{nameref}
\usepackage{graphicx}
\usepackage{float} % necessary for figures with placement [H]
\usepackage{xspace}
\usepackage{parskip}

\newcommand{\suser}[1]{\href{https://www.signaly.cz/#1}{\texttt{@#1}}}

\newcommand{\answercount}[1]{Odpovědělo \input{graphs/counts/#1.txt} respondentů}

\newcommand{\includegraph}[2]{
  \begin{figure}[H]
    \centering
    \textbf{#2}
    \includegraphics{graphs/img/#1.pdf}
    \answercount{#1}
  \end{figure}
}

\newcommand{\includegraphonly}[2]{
  \begin{figure}[H]
    \centering
    \textbf{#2}
    \includegraphics{graphs/img/#1.pdf}
  \end{figure}
}

\newcommand{\qtype}{\textbf{Typ otázky:}
}
\newcommand{\pickOne}{Výběr jedné možnosti z předdefinované sady\xspace}
\newcommand{\pickMultiple}{Výběr více možností z předdefinované sady\xspace}
\newcommand{\withOther}{s možností formulovat vlastní volnou odpověď (\uv{Jiné})\xspace}
\newcommand{\series}{(Řada otázek stejného typu se stejnými možnostmi odpovědi.)\xspace}
\newcommand{\freeEntry}{Volná odpověď (textové pole)}


\setcounter{secnumdepth}{3}
\setcounter{tocdepth}{3}

\extrafloats{200} % greater limit of floats awaiting placement

\bibliography{biblio}

\author{Jakub Pavlík}
\title{Vyhodnocení dotazníku\\ \uv{signaly.cz z pohledu uživatelů}}

\begin{document}

\setlength{\parindent}{0.5cm}

\maketitle

\section*{Úvod}

Signály letos slaví deset let provozu jako sociální síť.
Kromě toho, že se slaví a vzpomíná, však naléhavě vyvstávají otázky
ohledně dalšího směřování celého projektu:
není třeba provoz na Signálech pozorovat zvlášť dlouho ani pozorně,
aby pozorovatel pochopil, že heslo z titulní stránky
\emph{33 637 mladých křesťanů skrze signály.cz tvoří společenství}
se s realitou hrubě rozchází:
web sice snad má přes 33000 registrovaných
uživatelů, převážná většina uživatelských účtů však je buďto
úplně opuštěná, nebo vykazuje jen minimální aktivitu.
Navíc se ukazuje -- jak bude vidět i dále ve výstupech z dotazníku --
že uživatelskou základnu spíše než mladí křesťané
tvoří \emph{křesťané, kteří v době spuštění Signálů jako komunitního
  webu byli mladí:}
uživatelská základna stárne, Signály nejsou příliš úspěšné
v oslovování nových uživatelů z řad mládeže.
Mezi deklarovaným posláním a tím, kým a k čemu jsou Signály skutečně
využívány, zeje propast, která se nadále rozevírá.

Tyto skutečnosti si žádají nové definování vize a poslání Signálů.
Přispět k tomu jsem se pokusil s pomocí dotazníku, jehož vyhodnocení
překládám. V základech tohoto podniku stojí předpoklad,
který jsem formuloval již v úvodu k dotazníku:
\emph{hlavní hodnotou sociální sítě jsou její uživatelé --
  a cesta k uživatelům novým vede přes pochopení potřeb těch
  stávajících}.

Šetření probíhalo v týdnu 26.~7.--2.~8. 2017
s pomocí dotazníku postaveného v Google Forms.
K oslovení respondentů byl využit článek na osobním blogu
\url{http://dromedar.signaly.cz}, redakcí Signálů pro ten účel
na celý týden laskavě vystavený v bloku redakčního obsahu
na titulní stránce, takže se dostal i k těm, kdo daný blog normálně
nesledují.
Za týden, kdy byl dotazník propagován a probíhal sběr odpovědí,
se jich podařilo nasbírat 215. Je potřeba zohlednit to, že
šetření probíhalo v době dovolených, po poměrně krátkou dobu
a bez masivnější \uv{marketingové podpory}.
I přesto počet sebraných odpovědí jistě má určitou vypovídací
hodnotu ohledně velikosti aktivní části uživatelské základny.

Na následujících stránkách budou nejprve představeny otázky,
na které respondenti odpovídali, spolu s grafy prostého
rozložení odpovědí. Následně nabídneme jejich interpretaci.

\section{Data a zdrojové kódy}

Sebraná data, spolu s kompletními zdrojovými kódy pro sestavení
této publikace (sazba, skripty generující z dat grafy)
a historií jejího vzniku,
jsou na Githubu v repozitáři\\
\url{https://github.com/igneus/signaly-dotaznik-2017}.

\input{otazky}

\section{Vypovídací hodnota sebraných dat}

Než přikročíme k interpretaci sebraných dat, je třeba věnovat
pozornost skutečnostem, které mají vliv na jejich vypovídací
hodnotu.

\subsection{Pravdivost údajů}

Jako u většiny anonymních internetových dotazníků,
nemáme žádnou jistotu, že respondenti skutečně patří do cílové
skupiny. Dotazník mohl vyplnit každý, kdo se o něm dozvěděl.

Nebylo vyvinuto žádné úsilí ve snaze zajistit, aby každý
respondent mohl dotazník vyplnit právě jednou.
Pokud by to tedy někoho bavilo, mohl dotazníků vyplnit
neomezené množství.

Z povahy dotazníku plyne nemožnost ověřit, zda respondenti
uváděli pravdivé informace.

\subsection{Reprezentativnost vzorku}

I pokud by všichni respondenti byli skutečně uživateli Signálů,
vyplnili dotazník každý právě jednou a pravdivě,
je nezbytné ptát se, nakolik je sebraný vzorek reprezentativní.
Jak velkou část aktivních uživatelů se podařilo podchytit?
Jsou ve vzorku zastoupeny pokud možno všechny specifické
podskupiny, nebo došlo k nějaké selekci?

Dotazník cílil především na aktivní uživatele, tedy na ty,
kdo Signály navštěvují denně nebo častěji. Tomu byla uzpůsobená
doba sběru dat, omezená na jeden týden,
a způsob oslovování respondentů, omezující se na
decentní propagaci na titulní stránce.
Pokud je nějaká významnější skupina uživatelů, která na Signály
chodí pravidelně, ale jen týdně nebo řídčeji, je pravděpodobné,
že ve vzorku není příliš zastoupena.

Šetření probíhalo na přelomu července a srpna, tedy v době
dovolených, kdy lze na sociálních sítích předpokládat spíše
nižší aktivitu uživatelů.
Lze předpokládat, že to mělo vliv na \emph{velikost} vzorku,
který by v jiném ročním období snad byl o něco větší.
Výrazný vliv na jeho \emph{složení} nepředpokládám.

K určité selekci respondentů mohlo dojít vzhledem k formě propagace.
Dotazník byl totiž propagován hlavně článkem zařazeným v bloku
redakčního obsahu na titulní stránce. Redakční obsah je sice
na prominentním místě, všem na očích, ale považuji za pravděpodobné,
že mu i značná část uživatelů věnuje pozornost malou až žádnou.

Dalším možným faktorem selekce je osoba průzkum provádějící:
od začátku totiž šlo o soukromý osobní počin. Autor se pak stěží
může považovat za osobu známou a všeobecně oblíbenou.
Je tudíž možné, že jsou ve vzorku nedostatečně zastoupeni
\uv{ti, kdo nemají rádi \suser{dromedar}a} (a tudíž nemají chuť
zapojovat se do jeho podniků).
Není ale důvod myslet si, že by tato skupina, pokud vůbec existuje,
byla nějak velká nebo charakteristicky profilovaná a její případné
nezapojení se mělo zásadnější vliv na celkový obraz.

\subsection{Předpoklad spolehlivosti}

Viděli jsme, že spolehlivost sebraných dat ohrožuje celá řada
faktorů. Na druhou stranu se ale můžeme ptát:
je nějaký důvod domnívat se, že data v některém ohledu nespolehlivá
skutečně jsou?

Nevšiml jsem si ničeho, co by naznačovalo, že někdo dotazník
záměrně vyplnil opakovaně. Nevidím žádný důvod předpokládat
principielní nepravdivost sebraných dat.

Pokud jde o reprezentativnost vzorku, je namístě větší opatrnost.
Pro zpřesnění pohledu by bylo užitečné mít informace
o návštěvnosti webu ve sledovaném období a o skladbě (např. věkové)
aktivních uživatelů, jak je mají k dispozici
správci webu.\footnote{
  Nepamatuji si ovšem, že by data tohoto druhu za celou historii
  Signálů kdy byla zveřejněna, a předpokládám, že se z toho
  spíš nebude dělat výjimka.
}
I zde však snad můžeme předpokládat, že vzorek víceméně
reprezentativní je.

\textbf{Dále tedy budeme předpokládat, že sebraná data jsou v zásadě
pravdivá a představují reprezentativní vzorek uživatelské základny
Signálů.}


\section{Signály jako produkt}

\subsection{Kdo Signály používá}

Z grafů odpovědí na jednotlivé otázky je zřejmé,
že na Signálech výrazně převažují ženy nad muži (\ref{sec:pohlavi})
a nejsilnější věkovou skupinou (\ref{sec:vek}) je skupina
21-40 let. Typickým uživatelem je žena v produktivním věku.
To lze ilustrovat grafem zastoupení pohlaví v jednotlivých
věkových skupinách:

\includegraphonly{vek_x_pohlavi}{Věkové skupiny podle pohlaví}

Většina aktivních uživatelů Signály používá
(nebo tu přinejmenším má profil) opravdu dlouho (\ref{sec:mamprofil}).
Šest nebo více let jsou registrovány více než tři čtvrtiny uživatelů,
více než jedna třetina pak je registrována po celých deset let,
kdy Signály jako sociální síť fungují. Nebude tedy velkým
přeháněním, když řekneme, že Signály -- přinejmenším pokud
jde o uživatelskou základnu -- dodnes žijí ze vkladu prvních let
provozu.

Při pohledu zvenku se může zdát, že aktivní společenství
\uv{Ženy, matky, manželky}\footnote{
  \url{https://www.signaly.cz/zeny-matky-manzelky}},
cílící na ženy v produktivním věku, na Signály přitáhlo
velké množství nových uživatelek a početní převahu této skupiny
dále posílilo. Sebraná data ale tuto domněnku nepotvrzují.
Naopak se zdá, že v poslední době na Signálech žen nijak
dramaticky nepřibývá a ŽMM tedy zřejmě bere členky především
z řad stávajících uživatelek.
(Popř. se zástupy čerstvých \uv{uživatelek-matek-manželek} do výsledků
průzkumu z nějakého důvodu nepromítly, což by ovšem kladlo
velký otazník před náš předpoklad reprezentativnosti sebraných dat.)

\includegraphonly{profil_x_pohlavi}{Stáří uživatelského profilu a pohlaví}

Další nápadnou charakteristikou uživatelů Signálů je poměrně
vysoká míra vzdělanosti. Více než polovinu uživatelů,
kteří mají ukončené vzdělání, tvoří absolventi vysokých
škol.
Podíl vysokoškoláků mezi uživateli Signálů je tak několikanásobně
větší než v celku české populace, kde vysokoškoláci tvoří jen 12,5~\%,
v ročnících čerstvých absolventů pak až ke 20~\%.\footnote{
  Srovnáváno s údaji z výsledků sčítání lidu 2011
  \cite{sl2011vzdelani}
}

\includegraphonly{vzdelani_sloucene}{Dosažené vzdělání -- hrubý pohled}

\subsection{Používané funkcionality}

\subsection{\uv{Uživatelské úspěchy}}

\subsection{Signály v konkurenci}


\section{\uv{Sociální ovzduší}}

Je prostředí Signálů nevlídné? Odrazuje nové uživatele?
Je tu málo mladých z tohoto důvodu?
Vyhodnocovaný dotazník na tyto otázky nemůže přímo odpovědět,
protože případné odrazené uživatele nemohl zachytit. Cílil
na současné aktivní uživatele, tedy na ty, kdo i v tom -- možná
nevlídném -- prostředí vydrželi, našli si v něm své místo
a naučili se přežít. Zklamamé uživatele by bylo z podstaty věci
nutné oslovit jinak, než dotazníkem na Signálech. Třeba posílat
e-mail těm, kdo se na Signálech už dlouho neukázali.
Na druhou stranu, zkušenosti stávajících uživatelů
(sebrané, jak jsme viděli, mnohdy v průběhu dlouhé řady let)
mají i pro tyto otázky určitou vypovídací hodnotu.

Série otázek \uv{Těší mě ...} (\ref{sec:ostatniuzivatele})
naznačuje, že problémem v interakcích mezi uživateli
spíš \emph{není} nedostatečné soukromí, resp. to, že by interakce
s cizími lidmi byly obecně vnímány jako nepříjemné.
Výrazněji ambivalentně hodnoceným typem interakce je pouze
soukromá zpráva od cizího člověka.

Nezanedbatelná část uživatelů přiznává, že zažila nepříjemné
komentáře, popř. vzkazy. Jen malá část však kvůli těmto
zážitkům chtěla Signály opustit, nebo by byla pro vypovězení
\uv{pachatelů} ze Signálů. Nezdá se, že by se ze sebraných dat
daly dělat nějaké závěry ohledně dalšího vývoje Signálů po technické
stránce nebo potřebnosti agilnějšího postupu při \uv{správě komunity}.


\section{Sondy}

\section*{Poděkování}

Závěrem je třeba poděkovat všem, kdo se na dotazníku podíleli:
\suser{psycho-kat}, jejíž otázka po \uv{tvrdých datech}
vznik dotazníku vyprovokovala;
\suser{JiKu}, \suser{slu-nicko}, \suser{plihalik}, \suser{mia-maru},
\suser{Papo} a \suser{alweryon}, kteří pročetli otázky před spuštěním
dotazníku a poskytli mi k nim zpětnou vazbu;
šéfredaktorce \suser{Kollenka}, která článek s pozváním k vyplnění
dotazníku na celý týden umístila do bloku redakčního obsahu,
takže se dostal
k mnohem více uživatelům, než by běžný příspěvek z osobního blogu
kdy mohl;
zejména ale patří velký dík každému, kdo věnoval kus svého času
a dotazník vyplnil.

\tableofcontents

\end{document}
