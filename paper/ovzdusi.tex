\section{\uv{Sociální ovzduší}}

Je prostředí Signálů nevlídné? Odrazuje nové uživatele?
Je tu málo mladých z tohoto důvodu?
Vyhodnocovaný dotazník na tyto otázky nemůže přímo odpovědět,
protože případné odrazené uživatele nemohl zachytit. Cílil
na současné aktivní uživatele, tedy na ty, kdo i v tom -- možná
nevlídném -- prostředí vydrželi, našli si v něm své místo
a naučili se přežít. Zklamamé uživatele by bylo z podstaty věci
nutné oslovit jinak, než dotazníkem na Signálech. Třeba posílat
e-mail těm, kdo se na Signálech už dlouho neukázali.
Na druhou stranu, zkušenosti stávajících uživatelů
(sebrané, jak jsme viděli, mnohdy v průběhu dlouhé řady let)
mají i pro tyto otázky určitou vypovídací hodnotu.

Série otázek \uv{Těší mě ...} (\ref{sec:ostatniuzivatele})
naznačuje, že problémem v interakcích mezi uživateli
spíš \emph{není} nedostatečné soukromí, resp. to, že by interakce
s cizími lidmi byly obecně vnímány jako nepříjemné.
Výrazněji ambivalentně hodnoceným typem interakce je pouze
soukromá zpráva od cizího člověka.

Nezanedbatelná část uživatelů přiznává, že zažila nepříjemné
komentáře, popř. vzkazy. Jen malá část však kvůli těmto
zážitkům chtěla Signály opustit, nebo by byla pro vypovězení
\uv{pachatelů} ze Signálů. Nezdá se, že by se ze sebraných dat
daly dělat nějaké závěry ohledně dalšího vývoje Signálů po technické
stránce nebo potřebnosti agilnějšího postupu při \uv{správě komunity}.
