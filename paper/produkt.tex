\section{Signály jako produkt}

\subsection{Kdo Signály používá}

Z grafů odpovědí na jednotlivé otázky je zřejmé,
že na Signálech výrazně převažují ženy nad muži (\ref{sec:pohlavi})
a nejsilnější věkovou skupinou (\ref{sec:vek}) je skupina
21-40 let. Typickým uživatelem je žena v produktivním věku.
To lze ilustrovat grafem zastoupení pohlaví v jednotlivých
věkových skupinách:

\includegraphonly{vek_x_pohlavi}{Věkové skupiny podle pohlaví}

Většina aktivních uživatelů Signály používá
(nebo tu přinejmenším má profil) opravdu dlouho (\ref{sec:mamprofil}).
Šest nebo více let jsou registrovány více než tři čtvrtiny uživatelů,
více než jedna třetina pak je registrována po celých deset let,
kdy Signály jako sociální síť fungují. Nebude tedy velkým
přeháněním, když řekneme, že Signály -- přinejmenším pokud
jde o uživatelskou základnu -- dodnes žijí ze vkladu prvních let
provozu.

Při pohledu zvenku se může zdát, že aktivní společenství
\uv{Ženy, matky, manželky}\footnote{
  \url{https://www.signaly.cz/zeny-matky-manzelky}},
cílící na ženy v produktivním věku, na Signály přitáhlo
velké množství nových uživatelek a početní převahu této skupiny
dále posílilo. Sebraná data ale tuto domněnku nepotvrzují.
Naopak se zdá, že v poslední době na Signálech žen nijak
dramaticky nepřibývá a ŽMM tedy zřejmě bere členky především
z řad stávajících uživatelek.
(Popř. se zástupy čerstvých \uv{uživatelek-matek-manželek} do výsledků
průzkumu z nějakého důvodu nepromítly, což by ovšem kladlo
velký otazník před náš předpoklad reprezentativnosti sebraných dat.)

\includegraphonly{profil_x_pohlavi}{Stáří uživatelského profilu a pohlaví}

Další nápadnou charakteristikou uživatelů Signálů je poměrně
vysoká míra vzdělanosti. Více než polovinu uživatelů,
kteří mají ukončené vzdělání, tvoří absolventi vysokých
škol.
Podíl vysokoškoláků mezi uživateli Signálů je tak několikanásobně
větší než v celku české populace, kde vysokoškoláci tvoří jen 12,5~\%,
v ročnících čerstvých absolventů pak až ke 20~\%.\footnote{
  Srovnáváno s údaji z výsledků sčítání lidu 2011
  \cite{sl2011vzdelani}
}

\includegraphonly{vzdelani_sloucene}{Dosažené vzdělání -- hrubý pohled}

\subsection{Používané funkcionality}

\subsection{\uv{Uživatelské úspěchy}}

\subsection{Signály v konkurenci}
