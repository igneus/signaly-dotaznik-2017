\section{Signály jako produkt}

\subsection{Kdo Signály používá}

Z grafů odpovědí na jednotlivé otázky je zřejmé,
že na Signálech výrazně převažují ženy nad muži (\ref{sec:pohlavi})
a nejsilnější věkovou skupinou (\ref{sec:vek}) je skupina
21-40 let. Typickým uživatelem je žena v produktivním věku.
To lze ilustrovat grafem zastoupení pohlaví v jednotlivých
věkových skupinách:

\includegraphonly{vek_x_pohlavi}{Věkové skupiny podle pohlaví}

Většina aktivních uživatelů Signály používá
(nebo tu přinejmenším má profil) opravdu dlouho (\ref{sec:mamprofil}).
Šest nebo více let jsou registrovány více než tři čtvrtiny uživatelů,
více než jedna třetina pak je registrována po celých deset let,
kdy Signály jako sociální síť fungují. Nebude tedy velkým
přeháněním, když řekneme, že Signály -- přinejmenším pokud
jde o uživatelskou základnu -- dodnes žijí ze vkladu prvních let
provozu.

Při pohledu zvenku se může zdát, že aktivní společenství
\uv{Ženy, matky, manželky}\footnote{
  \url{https://www.signaly.cz/zeny-matky-manzelky}},
cílící na ženy v produktivním věku, na Signály přitáhlo
velké množství nových uživatelek a početní převahu této skupiny
dále posílilo. Sebraná data ale tuto domněnku nepotvrzují.
Naopak se zdá, že v poslední době na Signálech žen nijak
dramaticky nepřibývá a ŽMM tedy zřejmě bere členky především
z řad stávajících uživatelek.
(Popř. se zástupy čerstvých \uv{uživatelek-matek-manželek} do výsledků
průzkumu z nějakého důvodu nepromítly, což by ovšem kladlo
velký otazník před náš předpoklad reprezentativnosti sebraných dat.)

\includegraphonly{profil_x_pohlavi}{Stáří uživatelského profilu a pohlaví}

Další nápadnou charakteristikou uživatelů Signálů je poměrně
vysoká míra vzdělanosti. Více než polovinu uživatelů,
kteří mají ukončené vzdělání, tvoří absolventi vysokých
škol.
Podíl vysokoškoláků mezi uživateli Signálů je tak několikanásobně
větší než v celku české populace, kde vysokoškoláci tvoří jen 12,5~\%,
v ročnících čerstvých absolventů pak až ke 20~\%.\footnote{
  Srovnáváno s údaji z výsledků sčítání lidu 2011
  \cite{sl2011vzdelani}
}

\includegraphonly{vzdelani_sloucene}{Dosažené vzdělání -- hrubý pohled}

\subsection{Používané funkcionality}

Srovnání odpovědí na \uv{hodnotové} otázky (\ref{sec:kcemu})
s odpověďmi na otázky po frekvenci využívání funkcionalit
(\ref{sec:funkcionality}) ukazuje, že to, že je nějaká služba
pro uživatele důležitá, neznamená, že ji používá často.
Více než polovina respondentů uvedla, že
na Signály chodí hlavně proto, že rádi čtou blogy,
a podobně i v přehledu \uv{uživatelských úspěchů}
jasně dominují \uv{úspěchy} čtenářské.
Signály se tím jasně profilují jako \textbf{blogovací platforma}.
U otázek na frekvenci čtení blogů však převládají odpovědi
\uv{alespoň jednou týdně} a \uv{méně než jednou týdně}.
Každodenních čtenářů je hrstka.

Po čtení blogů uživatelé používají Signály hlavně
ke \uv{kontaktu s kamarády}, tedy jako \textbf{sociální síť}.
Mezi třemi nabízenými skupinami kamarádů převládají ti,
které uživatelé znají právě jen ze Signálů.

Dalšími pro uživatele důležitými funkcionalitami jsou přehled
křesťanských akcí, možnost diskutovat (od nasazení verze JP2
až donedávan odkázaná na komentáře pod blogy a příspěvky na zdech
uživatelů a společenství) a redakční obsah.

Výše načrtnutý obraz potvrzují i odpovědi na otázku
po \uv{uživatelských úspěších} (\ref{sec:uspechy}).
Dominují úspěchy čtenářské a autorské (vlastní Signálům jakožto
\textbf{blogovací platformě}),
následují úspěchy vlastní \textbf{sociální síti}
a další.

\subsection{Signály v konkurenci}

Z ostatních sociálních sítí uživatelé Signálů nejčastěji používají --
nepřekvapivě -- Facebook (\ref{sec:jinesite}).

Srovnání, jak často používají Facebook aktivní a méně aktivní
uživatelé Signálů, ukazuje, že mezi aktivitou na Facebooku
a na Signálech není žádný vztah. Používanost Facebooku je prakticky
totožná mezi těmi, kdo na Signály chodí denně,
jako mezi těmi, kdo sem zavítají méně často.

\includegraphonly{jine_site_Facebook}{Jak často navštěvuješ Facebook -- všichni}

\includegraphonly{nejaktivnejsi_signalnici_a_fb}{Jak často navštěvuješ Facebook -- ti, kdo chodí na Signály denně}

\includegraphonly{mene_aktivni_signalnici_a_fb}{Jak často navštěvuješ Facebook -- ti, kdo chodí na Signály méně často}

Odpovědi na otázky, které zjišťují hodnotu Signálů v konkurenci
dalších sociálních sítí (\ref{sec:konkurence}),
ukazují, že silou Signálů jsou zejména obsah a aktivity.
Zřejmě není až tak důležité, \emph{kdo} na Signálech je,
ale \emph{co} tu tvoří, resp. co \emph{se tu děje}.
